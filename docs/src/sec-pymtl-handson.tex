%=========================================================================
% Hands-On: RegIncr
%=========================================================================

\stepcounter{taskseccount}
\section[{\it Hands-On} Max/RegIncr]{}
\scheduleslide{6}

\begin{frame}{\IT{Hands-On:} PyMTL Basics with Max/RegIncr}
\begin{cbxlist}
  \1 Task 2.1: Experiment with \TT{Bits}
  \1 Task 2.2: Interactively simulate a max unit
  \1 Task 2.3: Write a registered incrementer (RegIncr) model
  \1 Task 2.4: Test the RegIncr model
  \1 Task 2.5: Translate the RegIncr model into Verilog
  \1 Task 2.6: Simulate the RegIncr model with line tracing
  \1 Task 2.7: Simulate the RegIncr model with VCD dumping
  \1 Task 2.8: Compose a pipeline with two RegIncr models
  \1 Task 2.9: Compose a pipeline with N RegIncr models
  \1 Task 2.10: Parameterize test to verify multiple Ns
\end{cbxlist}
\end{frame}

\begin{frame}{\IT{Hands-On:} PyMTL Basics with Max/RegIncr}
\begin{cbxlist}
  \1 \BF{Task 2.1: Experiment with \TT{Bits}}
  \1 \BF{Task 2.2: Interactively simulate a max unit}
  \1 Task 2.3: Write a registered incrementer (RegIncr) model
  \1 Task 2.4: Test the RegIncr model
  \1 Task 2.5: Translate the RegIncr model into Verilog
  \1 Task 2.6: Simulate the RegIncr model with line tracing
  \1 Task 2.7: Simulate the RegIncr model with VCD dumping
  \1 Task 2.8: Compose a pipeline with two RegIncr models
  \1 Task 2.9: Compose a pipeline with N RegIncr models
  \1 Task 2.10: Parameterize test to verify multiple Ns
\end{cbxlist}
\end{frame}

%-------------------------------------------------------------------------
\begin{frame}[fragile]{\TT{Bits} Class for Fixed-Bitwidth Values}
%-------------------------------------------------------------------------
\colorbox{gray!30!white}{\parbox{\tw}{\rule[-0.4em]{0pt}{1.4em}\centering\textbf{%
  PyMTL Bits Operators
}}}

{\scriptsize

\newenvironment{optbl2}[1]
{
  %\vspace{0.5em}
  \centering{\BF{\footnotesize{{#1}}}}
  \smallskip

  %\hspace{0.75em}
  \begin{tabular}{>{\ttfamily\centering\arraybackslash}p{0.23\tw}p{.76\tw}}
}{
  \end{tabular}
}

\newenvironment{optbl}[1]
{
  %\vspace{0.5em}
  \centering{\BF{\footnotesize{{#1}}}}
  \smallskip

  %\hspace{0.75em}
  \begin{tabular}{>{\ttfamily\centering\arraybackslash}p{0.33\tw}p{.66\tw}}
%\toprule
}{
  \end{tabular}
}

\begin{cbxcols}
\begin{column}{0.30\tw}
\vspace{.3in}

\begin{optbl2}{Logical Operators}
  \verb|&|   & bitwise AND   \\
  |          & bitwise OR    \\
  \verb|^|   & bitwise XOR   \\
  \verb|^~|  & bitwise XNOR  \\
  \verb|~|   & bitwise NOT   \\
\end{optbl2}

\vspace{.15in}
\begin{optbl2}{Arith. Operators}
  \verb|+|   & addition         \\
  \verb|-|   & subtraction      \\
  \verb|*|   & multiplication   \\
  \verb|/|   & division         \\
  \verb|%|   & modulo           \\
\end{optbl2}


\end{column}
\begin{column}{0.30\tw}
\vspace{.3in}

\begin{optbl}{Shift Operators}
  \verb|>>|  & shift right            \\
  \verb|<<|  & shift left             \\
\end{optbl}

\vspace{.15in}

\begin{optbl}{Slice Operators}
  \verb|[x]|   & get/set bit x            \\
  \verb|[x:y]| & get/set bits             \\
               &  x upto y                \\

\end{optbl}

\vspace{.15in}

\begin{optbl}{Reduction Operators}
  \verb|reduce_and| & reduce via AND \\
  \verb|reduce_or|  & reduce via OR  \\
  \verb|reduce_xor| & reduce via XOR \\
\end{optbl}

\end{column}
\begin{column}{0.40\tw}
\vspace{.3in}

\begin{optbl2}{Relational Operators}
  \verb|==|  & equal                  \\
  \verb|!=|  & not equal              \\
  \verb|>|   & greater than           \\
  \verb|>=|  & greater than or equals \\
  \verb|<|   & less than              \\
  \verb|<=|  & less than or equals    \\
\end{optbl2}

\vspace{.15in}

\begin{optbl2}{Other Functions}
  \verb|concat| & concatenate         \\
  \verb|sext|   & sign-extension      \\
  \verb|zext|   & zero-extension      \\
\end{optbl2}

\end{column}
\end{cbxcols}

}

\end{frame}

%-------------------------------------------------------------------------
\begin{task}\begin{frame}[fragile]{Experiment with \TT{Bits}}
%-------------------------------------------------------------------------

\vspace{-0.15in}
\begin{cbxcols}

\begin{column}{0.4\tw}
\begin{lstlisting}[xleftmargin={0in},numbers={none},basicstyle={\ttfamily\small}]
 % cd \midtilde
 % ipython

 >>> from pymtl import *
 >>> a = Bits( 8, 5 )
 >>> b = Bits( 8, 3 )
 >>> a + b
 Bits( 8, 0x08 )
 >>> a - b
 Bits( 8, 0x02 )
 >>> a | b
 Bits( 8, 0x07 )
 >>> a & b
 Bits( 8, 0x01 )
\end{lstlisting}
\end{column}

\begin{column}{0.4\tw}
\begin{lstlisting}[xleftmargin={0in},numbers={none},basicstyle={\ttfamily\small}]



 >>> c = concat( a, b )
 >>> c
 Bits( 16, 0x0503 )
 >>> c[0:8]
 Bits( 8, 0x03 )
 >>> c[8:16]
 Bits( 8, 0x05 )
 >>> exit()
\end{lstlisting}
\end{column}

\end{cbxcols}
\end{frame}
\end{task}

%-------------------------------------------------------------------------
\begin{task}\begin{frame}[fragile]{Interactively simulate a max unit}
%-------------------------------------------------------------------------

\vspace{-0.15in}
\begin{lstlisting}[xleftmargin={0in},numbers={none},basicstyle={\ttfamily\small}]
 % cd \midtilde/pymtl-tut/maxunit
 % ipython

 >>> from pymtl import *
 >>> from MaxUnitFL import MaxUnitFL
 >>> model = MaxUnitFL( nbits=8, nports=3 )
 >>> model.elaborate()
 >>> sim = SimulationTool(model)
 >>> sim.reset()
 >>> model.in_[0].value = 2
 >>> model.in_[1].value = 5
 >>> model.in_[2].value = 3
 >>> sim.cycle()
 >>> model.out
 Bits( 8, 0x05 )
 >>> exit()
\end{lstlisting}

\vspace{-1.7in}\hspace*{2.5in}
\cbxfigc[0.48\tw]{tut3-maxunit.svg.pdf}

\end{frame}
\end{task}

%-------------------------------------------------------------------------
%\begin{task}\begin{frame}[fragile]{Run RegIncrFL and RegIncrRTL tests}
%-------------------------------------------------------------------------
% \begin{verbatim}
%   % mkdir ~/pymtl-tut/build
%   % cd      ~/pymtl-tut/build
% \end{verbatim}
%
% \begin{verbatim}
%   % py.test ../regincr/RegIncrFL_test.py --verbose
%   % py.test ../regincr/RegIncrRTL_test.py --verbose
% \end{verbatim}
%
% \begin{centering}
% \TT{RegIncrFL\_test.py} should pass. \\*
% \TT{RegIncrRTL\_test.py} should \BF{fail}!
%
% \vspace{0.4in}
% We haven't implemented it yet!
%
% \end{centering}
%
% \end{frame}
% \end{task}

\begin{frame}{\IT{Hands-On:} PyMTL Basics with Max/RegIncr}
\begin{cbxlist}
  \1 Task 2.1: Experiment with \TT{Bits}
  \1 Task 2.2: Interactively simulate a max unit
  \1 \BF{Task 2.3: Write a registered incrementer (RegIncr) model}
  \1 \BF{Task 2.4: Test the RegIncr model}
  \1 \BF{Task 2.5: Translate the RegIncr model into Verilog}
  \1 Task 2.6: Simulate the RegIncr model with line tracing
  \1 Task 2.7: Simulate the RegIncr model with VCD dumping
  \1 Task 2.8: Compose a pipeline with two RegIncr models
  \1 Task 2.9: Compose a pipeline with N RegIncr models
  \1 Task 2.10: Parameterize test to verify multiple Ns
\end{cbxlist}
\end{frame}

%-------------------------------------------------------------------------
\begin{task}\begin{frame}[fragile]{Write a registered incrementer model}
%-------------------------------------------------------------------------

\vspace{-0.15in}
\begin{Verbatim}[commandchars=\\\{\}]
 % cd \midtilde/pymtl-tut/build
 % gedit ../regincr/RegIncrRTL.py
\end{Verbatim}
\vspace{-0.2in}

\lstinputlisting[xleftmargin={0.38in},firstline=8,lastline=21,firstnumber=8]%
{../../regincr_soln/RegIncrRTL.py}

\vspace{-0.22in}
\begin{Verbatim}
 % py.test ../regincr/RegIncrRTL_test.py --verbose
\end{Verbatim}

\vspace{-2.5in}\hspace*{2.4in}
\cbxfigc[0.45\tw]{tut3-regincr.svg.pdf}

\end{frame}
\end{task}

%-------------------------------------------------------------------------
\begin{task}\begin{frame}[fragile]{Test the RegIncr model}
%-------------------------------------------------------------------------

\vspace{-0.15in}
\begin{Verbatim}[commandchars=\\\{\}]
 % cd \midtilde/pymtl-tut/build
 % py.test ../regincr/RegIncrRTL_test.py --verbose
\end{Verbatim}
\smallskip
\begin{Verbatim}
 =============== test session starts ================
 platform darwin -- Python 2.7.5 -- pytest-2.6.4
 plugins: xdist
 collected 2 items

 ../regincr/RegIncrRTL_test.py::test_simple[8] PASSED
 ../regincr/RegIncrRTL_test.py::test_random[8] PASSED

 ============= 2 passed in 0.36 seconds =============
\end{Verbatim}

\end{frame}
\end{task}

%-------------------------------------------------------------------------
\begin{frame}{Translating PyMTL RTL into Verilog}
%-------------------------------------------------------------------------

\begin{cbxlist}

  \1 PyMTL models written at the register-transfer level of abstraction
     can be translated into Verilog source using the \TT{TranslationTool}

  \1 Generated Verilog can be used with commercial EDA toolflows to
    characterize area, energy, and timing

\end{cbxlist}

\vspace{0.3in}
\cbxfigc{pymtl-tut-refine.pdf}

\end{frame}

%-------------------------------------------------------------------------
\begin{frame}{Translation as Part of PyMTL SimJIT-RTL}
%-------------------------------------------------------------------------

  \vspace{0.2in}
  \cbxfigc<1\h0>{pymtl-tut-simjit0.pdf}
  \cbxfigc<2\h1>{pymtl-tut-simjit1.pdf}

\end{frame}

%-------------------------------------------------------------------------
\begin{frame}{PyMTL to Verilog Translation Limitations}
%-------------------------------------------------------------------------

{}The \BF{TranslationTool} has limitations on what it can translate:

\medskip
\begin{cbxlist}[ll]

  \1 \BF{Static elaboration} can use arbitratry Python \\ (connections
        $\Rightarrow$ connectivity graph $\Rightarrow$ structural
        Verilog)

  \1 \BF{Concurrent logic blocks} must abide by language restrictions

  \pause

    \2 Data must be communicated in/out/between blocks using \BF{signals} \\
       (InPorts, OutPorts, and Wires)

    \2 Signals may only contain \BF{bit-specific value types} \\
       (Bits/BitStructs)

    \2 Only pre-defined, \BF{translatable operators/functions} may be
       used \\ (no user-defined operators or functions)

    \2 Any variables that don't refer to signals must be \BF{integer
       constants}

\end{cbxlist}
\end{frame}

%-------------------------------------------------------------------------
\begin{task}\begin{frame}[fragile]{Translate the RegIncr model into Verilog}
%-------------------------------------------------------------------------

\vspace{-0.15in}
\begin{Verbatim}[commandchars=\\\{\}]
 % cd \midtilde/pymtl-tut/build
 % gedit ../regincr/RegIncrRTL_test.py
\end{Verbatim}
\vspace{-0.2in}

\lstinputlisting[xleftmargin={0.38in},firstline=20,lastline=29,firstnumber=20]%
{../../regincr_soln/RegIncrRTL_test.py}

\vspace{-0.22in}
\begin{Verbatim}[commandchars=\\\{\}]
 % py.test ../regincr/RegIncrRTL_test.py -v --test-verilog
 % gedit ./RegIncrRTL_*.v
\end{Verbatim}

\end{frame}
\end{task}

%-------------------------------------------------------------------------
\begin{frame}[fragile]{Example Verilog Generated from Translation}
%-------------------------------------------------------------------------

\vspace{-0.25in}
\begin{cbxcols}

\begin{column}{0.5\tw}
\lstinputlisting[language=verilog,xleftmargin={0.2in},firstline=4,lastline=25,firstnumber=4,basicstyle={\ttfamily\scriptsize}]%
{../../regincr_soln/RegIncrRTL_0x7a355c5a216e72a4.v}
\end{column}

\begin{column}{0.5\tw}
\lstinputlisting[language=verilog,xleftmargin={0.2in},firstline=26,lastline=50,firstnumber=26,basicstyle={\ttfamily\scriptsize}]%
{../../regincr_soln/RegIncrRTL_0x7a355c5a216e72a4.v}
\end{column}

\end{cbxcols}
\end{frame}

\begin{frame}{\IT{Hands-On:} PyMTL Basics with Max/RegIncr}
\begin{cbxlist}
  \1 Task 2.1: Experiment with \TT{Bits}
  \1 Task 2.2: Interactively simulate a max unit
  \1 Task 2.3: Write a registered incrementer (RegIncr) model
  \1 Task 2.4: Test the RegIncr model
  \1 Task 2.5: Translate the RegIncr model into Verilog
  \1 \BF{Task 2.6: Simulate the RegIncr model with line tracing}
  \1 \BF{Task 2.7: Simulate the RegIncr model with VCD dumping}
  \1 Task 2.8: Compose a pipeline with two RegIncr models
  \1 Task 2.9: Compose a pipeline with N RegIncr models
  \1 Task 2.10: Parameterize test to verify multiple Ns
\end{cbxlist}
\end{frame}

%-------------------------------------------------------------------------
\begin{frame}{Unit Tests vs. Simulators}
%-------------------------------------------------------------------------

\BF{Unit Tests:} \TT{ModelName\_tests.py}
\begin{itemize}
  \item Tests that verify the simulation behavior of a model isolation
  \item Test functions are executed by the \TT{py.test} testing framework
  \item Unit tests should always be written before simulator scripts!
\end{itemize}

\vspace{0.2in}

\BF{Simulators:} \TT{model-name-sim.py}
\begin{itemize}
  \item Simulators are meant for model evaluation and stats collection
  \item Simulation scripts take commandline arguments for configuration
  \item Used for experimentation and (design space) exploration!
\end{itemize}

\end{frame}

%-------------------------------------------------------------------------
\begin{task}\begin{frame}[fragile]{Simulate RegIncr with line tracing}
%-------------------------------------------------------------------------

\vspace{-0.15in}
\begin{Verbatim}[commandchars=\\\{\}]
 % cd \midtilde/pymtl-tut/build
 % python ../regincr/reg-incr-sim.py 10
 % python ../regincr/reg-incr-sim.py 10 --trace
 % python ../regincr/reg-incr-sim.py 20 --trace

   0: 04e5f14d (00000000) 00000000
   1: 7839d4fc (04e5f14d) 04e5f14e
   2: 996ab63d (7839d4fc) 7839d4fd
   3: 6d146dfc (996ab63d) 996ab63e
   4: 9cb87fec (6d146dfc) 6d146dfd
   5: ba43a338 (9cb87fec) 9cb87fed
   6: a0c394ff (ba43a338) ba43a339
   7: f72041ee (a0c394ff) a0c39500
   ...
\end{Verbatim}

\end{frame}
\end{task}

%-------------------------------------------------------------------------
\begin{frame}{Line Tracing vs. VCD Dumping}
%-------------------------------------------------------------------------

\begin{cbxlist}

  \1 \BF{Line Tracing}

    \2 Shows a single cycle per line and uses text characters to indicate
       state and how data moves through a system

    \vspace{0.08in}
    \2 Provides a way to visualize the high-level behavior of a system
       (e.g., pipeline diagrams, transaction diagrams)

    \vspace{0.08in}
    \2 Enables quickly debugging high-level functionality and performance
       bugs at the commandline

    \vspace{0.08in}
    \2 Can be used for FL, CL, and RTL models

  \1 \BF{VCD Dumping}

    \2 Captures the bit-level activity of every signal on every cycle

    \2 Requires a separate waveform viewer to visualize the signals

    \2 Provides a much more detailed view of a design

    \2 Mostly used for RTL models

\end{cbxlist}
\end{frame}

%-------------------------------------------------------------------------
\begin{task}\begin{frame}[fragile]{Simulate RegIncr with VCD dumping}
%-------------------------------------------------------------------------

\vspace{-0.15in}
\begin{Verbatim}[commandchars=\\\{\}]
 % cd \midtilde/pymtl-tut/build
 % python ../regincr/reg-incr-sim.py 10 --dump-vcd
 % gtkwave ./reg-incr-rtl-10.vcd
\end{Verbatim}

  \smallskip
  \cbxfigc{gtkwave-regincr.png}

\end{frame}
\end{task}

\begin{frame}{\IT{Hands-On:} PyMTL Basics with Max/RegIncr}
\begin{cbxlist}
  \1 Task 2.1: Experiment with \TT{Bits}
  \1 Task 2.2: Interactively simulate a max unit
  \1 Task 2.3: Write a registered incrementer (RegIncr) model
  \1 Task 2.4: Test the RegIncr model
  \1 Task 2.5: Translate the RegIncr model into Verilog
  \1 Task 2.6: Simulate the RegIncr model with line tracing
  \1 Task 2.7: Simulate the RegIncr model with VCD dumping
  \1 \BF{Task 2.8: Compose a pipeline with two RegIncr models}
  \1 \BF{Task 2.9: Compose a pipeline with N RegIncr models}
  \1 \BF{Task 2.10: Parameterize test to verify multiple Ns}
\end{cbxlist}
\end{frame}

%-------------------------------------------------------------------------
\begin{frame}{Structural Composition in PyMTL}
%-------------------------------------------------------------------------

\medskip
\begin{cbxcols}
\begin{column}{0.9\tw}
\begin{cbxlist}

  \1 In PyMTL, more complex designs can be created by hierarchically
     composing models using structural composition

  \1 Models are structurally composed by connecting their ports
     using \TT{s.connect()} or \TT{s.connect\_pairs()} statements

  \1 Data is communicated between PyMTL models using \TT{InPorts}
     and \TT{OutPorts}, not via method calls!

\end{cbxlist}
\end{column}
\end{cbxcols}
\end{frame}

%-------------------------------------------------------------------------
\begin{task}\begin{frame}[fragile]{Compose a pipeline with two RegIncrs}
%-------------------------------------------------------------------------

\vspace{-0.15in}
\begin{Verbatim}[commandchars=\\\{\}]
 % cd \midtilde/pymtl-tut/build
 % gedit ../regincr/RegIncrPipeline.py
\end{Verbatim}
\vspace{-0.2in}

\lstinputlisting[xleftmargin={0.38in},firstline=9,lastline=21,firstnumber=9]%
{../../regincr_soln/RegIncrPipeline.py}

\vspace{-0.22in}
\begin{Verbatim}
 % py.test ../regincr/RegIncrPipeline_test.py -sv
\end{Verbatim}

\vspace{-2.35in}\hspace*{2.5in}
\cbxfigc[0.48\tw]{tut3-regincr-2stage.svg.pdf}

\end{frame}
\end{task}

\begin{frame}[fragile]{Line Tracing from Pipelined RegIncr}

\begin{Verbatim}
  ../regincr/RegIncrPipeline_test.py::test_simple
    0: 04 (00 00) 00
    1: 06 (05 02) 02
    2: 02 (07 06) 06
    3: 0f (03 08) 08
    4: 08 (10 04) 04
    5: 00 (09 11) 11
    6: 0a (01 0a) 0a
    7: 0a (0b 02) 02
    8: 0a (0b 0c) 0c
  PASSED
\end{Verbatim}

\end{frame}

%-------------------------------------------------------------------------
\begin{frame}{Parameterizing Models in PyMTL}
%-------------------------------------------------------------------------

\medskip
\begin{cbxcols}
\begin{column}{0.9\tw}
\begin{cbxlist}

  \1 Static elaboration code (everything inside \TT{\_\_init\_\_} that is
     not in a decorated function) can use the full expressiveness of
     Python

  \1 Static elaboration code constructs a connectivity graph of
     components, is always Verilog translatable \\
     (as long as leaf modules are translatable)

  \1 Enables the creation of powerful and highly-parameterizable
     hardware generators

\end{cbxlist}
\end{column}
\end{cbxcols}
\end{frame}

%-------------------------------------------------------------------------
\begin{task}\begin{frame}[fragile]{Compose a pipeline with N RegIncrs}
%-------------------------------------------------------------------------

\vspace{-0.15in}
\begin{Verbatim}[commandchars=\\\{\}]
 % cd \midtilde/pymtl-tut/build
 % gedit ../regincr/RegIncrParamPipeline.py
\end{Verbatim}
\vspace{-0.2in}

\lstinputlisting[xleftmargin={0.38in},firstline=9,lastline=22,firstnumber=9]%
{../../regincr/RegIncrParamPipeline.py}

\vspace{-0.22in}
\begin{Verbatim}
 % py.test ../regincr/RegIncrParamPipeline_test.py -sv
\end{Verbatim}

\vspace{-2.35in}\hspace*{2.8in}
\cbxfigc[0.42\tw]{tut3-regincr-Nstage.svg.pdf}

\end{frame}
\end{task}

%-------------------------------------------------------------------------
\begin{frame}{Parameterizing Tests in PyMTL}
%-------------------------------------------------------------------------

\medskip
\begin{cbxcols}
\begin{column}{0.9\tw}
\begin{cbxlist}

  \1 We leverage the opensource \TT{py.test} package to drive test
     collection and execution in PyMTL

  \1 Significantly simplifies process of writing unit tests, and enables
     functionality such as parallel/distributed test execution and
     coverage reporting via plugins

  \1 More importantly, \TT{py.test} has powerful facilities for writing
     extensive and highly parameterizable unit tests

  \1 One example: the \TT{@pytest.mark.parametrize} decorator

\end{cbxlist}
\end{column}
\end{cbxcols}
\end{frame}

%-------------------------------------------------------------------------
\begin{task}\begin{frame}[fragile]{Parameterize test to verify multiple Ns}
%-------------------------------------------------------------------------

\vspace{-0.15in}
\begin{Verbatim}[commandchars=\\\{\}]
 % cd \midtilde/pymtl-tut/build
 % gedit ../regincr/RegIncrParamPipeline_test.py
\end{Verbatim}
\vspace{-0.2in}

\lstinputlisting[xleftmargin={0.38in},firstline=43,lastline=51,firstnumber=43]%
{../../regincr_soln/RegIncrParamPipeline_test.py}

\vspace{-0.22in}
\begin{Verbatim}
 % py.test ../regincr/RegIncrParamPipeline_test.py -sv
\end{Verbatim}

\end{frame}
\end{task}

\begin{frame}[fragile]{Line Tracing from Pipelined RegIncr}

\begin{Verbatim}
../regincr/RegIncrParamPipeline_test.py::test_simple[5]
  0: 04 (00 00 00 00 00) 00
  1: 06 (05 02 02 02 02) 02
  2: 02 (07 06 03 03 03) 03
  3: 0f (03 08 07 04 04) 04
  4: 08 (10 04 09 08 05) 05
  5: 00 (09 11 05 0a 09) 09
  6: 0a (01 0a 12 06 0b) 0b
  7: 0e (0b 02 0b 13 07) 07
  8: 10 (0f 0c 03 0c 14) 14
  9: 0c (11 10 0d 04 0d) 0d
  ...
PASSED
\end{Verbatim}

\end{frame}

