%=========================================================================
% LaTeX Document
%=========================================================================

\documentclass[handout]{cbxslides}

% albs modules

\usepackage{svg}
\usepackage{py}

%=========================================================================
% pymtl-listings
%=========================================================================

\definecolor{dmlgreen}    {RGB}{51,  160,  44}
\definecolor{dmlblue}     {RGB}{31,  120, 180}
\definecolor{dmlred}      {RGB}{202,   0,  32}

\lstdefinestyle{simple}{%
  language=Python,
  numbers={none},
  basicstyle={\ttfamily},
  moredelim={[is][\underbar]{__}{__}}
}

\lstset
{%
  language=Python,%
  alsoletter={.},
  morekeywords=[2]{
    @pytest.mark.parametrize,
    @s.tick,
    @s.tick_fl,
    @s.tick_cl,
    @s.tick_rtl,
    @s.combinational,
    s.connect,
    s.connect_auto,
    s.connect_dict,
    s.connect_pairs,
    Bits,
    Wire,
    InPort,
    OutPort,
    SimulationTool,
    TranslationTool,
    Model
  },
  basicstyle={\ttfamily\footnotesize},%
  keywordstyle={\color{cbxgreenC}},%
  keywordstyle={[2]\color{cbxblueC}},%
  commentstyle={\color{cbxredC}},
  lineskip={-0.005in},%
  numbers={left},%
  numberstyle={\tiny},%
  xleftmargin={0.2in},%
  showstringspaces={false},%
  keepspaces={true},%
  upquote={true},%
  columns={fullflexible},%
  stringstyle={\color{brown}},%
}%



%-------------------------------------------------------------------------
% Front matter
%-------------------------------------------------------------------------

\title
{%
  PyMTL and Pydgin Tutorial \\
  \rule[0.4em]{0.75\paperwidth}{0.5pt} \\
  Python Frameworks for Highly Productive \\
  Computer Architecture Research
}

\author
{%
  Derek Lockhart, Berkin Ilbeyi, Christopher Batten
}

\name      {PyMTL/Pydgin Tutorial: Python Frameworks for Highly Productive Computer Architecture Research}
\abbrvinst {ISCA 2015}

\affiliation
{%
  Computer Systems Laboratory\\
  School of Electrical and Computer Engineering\\
  Cornell University\\[\baselineskip]
  42nd Int'l Symp. on Computer Architecture, June 2015
}

\headerwidth {0.95\paperwidth}
\finalslide  {125}

%-------------------------------------------------------------------------
% Schedule Slide
%-------------------------------------------------------------------------
% Argument is used to highlight specific row, with the row index starting
% from one.

\newcommand{\rownum}[2]
{%
  \ifthenelse{\equal{#1}{#2}}{\rowcolor{cbxredA}}{}%
}

\newcommand{\scheduleslide}[1]
{%

  \ifthenelse{\equal{#1}{2}}{\def\rowcolorA{cbxredA}}{\def\rowcolorA{white}}%
  \ifthenelse{\equal{#1}{3}}{\def\rowcolorB{cbxredA}}{\def\rowcolorB{white}}%
  \ifthenelse{\equal{#1}{4}}{\def\rowcolorC{cbxredA}}{\def\rowcolorC{white}}%
  \ifthenelse{\equal{#1}{5}}{\def\rowcolorD{cbxredA}}{\def\rowcolorD{white}}%
  \ifthenelse{\equal{#1}{6}}{\def\rowcolorE{cbxredA}}{\def\rowcolorE{white}}%
  \ifthenelse{\equal{#1}{7}}{\def\rowcolorF{cbxredA}}{\def\rowcolorF{white}}%
  \ifthenelse{\equal{#1}{8}}{\def\rowcolorG{cbxredA}}{\def\rowcolorG{white}}%
  \ifthenelse{\equal{#1}{9}}{\def\rowcolorH{cbxredA}}{\def\rowcolorH{white}}%

  \begin{frame}{PyMTL/Pydgin Tutorial Schedule}
  \renewcommand{\arraystretch}{1.25}
  \setlength{\tabcolsep}{2pt}

  \hspace*{-0.08in}
  \begin{tabular}{rcr<{\kern4\tabcolsep}@{}l}
   8:30am &--&  8:50am & Virtual Machine Installation and Setup               \\         \rowcolor{\rowcolorA}
   8:50am &--&  9:00am & \IT{Presentation:} PyMTL/Pydgin Tutorial Overview    \\\midrule \rowcolor{\rowcolorB}
   9:00am &--&  9:10am & \IT{Presentation:} Introduction to Pydgin            \\         \rowcolor{\rowcolorC}
   9:10am &--& 10:00am & \IT{Hands-On:} Adding a GCD Instruction using Pydgin \\         \rowcolor{\rowcolorD}
  10:00am &--& 10:10am & \IT{Presentation:} Introduction to PyMTL             \\         \rowcolor{\rowcolorE}
  10:10am &--& 11:00am & \IT{Hands-On:} PyMTL Basics with Max/RegIncr           \\\midrule \rowcolor{\rowcolorF}
  11:00am &--& 11:30am & Coffee Break                                         \\\midrule \rowcolor{\rowcolorG}
  11:30am &--& 11:40am & \IT{Presentation:} Multi-Level Modeling with PyMTL   \\         \rowcolor{\rowcolorH}
  11:40am &--& 12:30pm & \IT{Hands-On:} FL, CL, RTL Modeling of a GCD Unit    \\
  \end{tabular}
  \end{frame}
}

%-------------------------------------------------------------------------
% Document
%-------------------------------------------------------------------------

\begin{document}

\frame[t,plain]{\titlepage}

%=========================================================================
% Presentation: Overview
%=========================================================================

\section[{\it Presentation} Overview]{}

\begin{frame}[t]{Typical Research Methodologies: Application-Level}
\begin{cbxcols}

  \begin{column}{0.4\tw}
    \cbxfigc{cs-stack_stack_app_labels-split.svg.pdf}
  \end{column}

  \begin{column}{0.6\tw}
    \vspace{-0.05in}
    \begin{cbxlist}

      \1 General Approach
         \2 Use real machines
         \2 Use real machines with dynamic instrumentation (e.g., Pin)
         \2 Use fast instruction set simulators or emulators (e.g.,
            SimIt-ARM, QEMU)

      \1 Benefits
         \2 Fast execution enables experimenting \\ with large, realistic
            applications

      \1 Challenges
         \2 Difficult to explore applications for emerging architectures
            which do not \\ exist yet

    \end{cbxlist}
  \end{column}

\end{cbxcols}
\end{frame}

\begin{frame}[t]{Typical Research Methodologies: Architecture-Level}
\begin{cbxcols}

  \begin{column}{0.4\tw}
    \cbxfigc{cs-stack_stack_arch_labels-split.svg.pdf}
  \end{column}

  \begin{column}{0.6\tw}
    \vspace{-0.05in}
    \begin{cbxlist}

      \1 General Approach
         \2 Use standard benchmark suite (e.g., Splash2, PARSEC, Rodina)
         \2 Modify standard cycle-level C/C++ simulator (e.g., SESC,
            Simics, gem5)
         \2 Use standard high-level physical modeling tool (e.g., CACTI,
            Wattch, Orion, McPAT)

      \1 Benefits
         \2 More accurate than ISA simulation
         \2 Faster and more flexible design-space exploration than lower-level models

      \1 Challenges
         \2 Experimenting with large, realistic apps
         \2 Physical modeling of radically new arch

    \end{cbxlist}
  \end{column}

\end{cbxcols}
\end{frame}

\begin{frame}[t]{Typical Research Methodologies: VLSI-Level}
\begin{cbxcols}

  \begin{column}{0.4\tw}
    \cbxfigc{cs-stack_stack_vlsi_labels-split.svg.pdf}
  \end{column}

  \begin{column}{0.6\tw}
    \vspace{-0.05in}
    \begin{cbxlist}

      \1 General Approach
         \2 Possibly start with open-source IP (e.g., FabScalar,
            OpenRISC/SPARC, NetMaker)
         \2 Write small microbenchmarks or embedded applications
         \2 Implement SystemVerilog/Verilog/VHDL RTL (or Bluespec
            GAA) model of design
         \2 Use standard commercial ASIC CAD tools to estimate cycle
            time, area, energy

      \1 Benefits
         \2 More accurate physical characterization
         \2 Increases credibility of design

      \1 Challenges
         \2 Only small apps possible due to slow sims
         \2 Cumbersome design-space exploration

    \end{cbxlist}
  \end{column}

\end{cbxcols}
\end{frame}

\begin{frame}[t]{Vertically-Integrated Modeling Environment}
\begin{cbxcols}

  \begin{column}{0.51\tw}
    \cbxfigc{cs-stack-vint.svg.pdf}
  \end{column}

  \begin{column}{0.49\tw}
    \begin{cbxlist}

      \1 Unified package with integrated applications, test programs,
         cross compilers, proxy kernels, full OS kernels, ISA emulators,
         microarchitectural models, RTL models, ASIC/FPGA CAD scripts,
         and unit tests

      \1 Support for rapid/iterative design-space exploration across
         abstraction layers especially for emerging applications and
         radically new architectures

    \end{cbxlist}
  \end{column}

\end{cbxcols}
\end{frame}

\begin{frame}[t]{Highly Productive Modeling Environment}
  \cbxfig<1\h0>[0.75\tw]{hprod_0-split.svg.pdf}
  \cbxfig<2\h0>[0.75\tw]{hprod_0_1-split.svg.pdf}
  \cbxfig<3\h0>[0.75\tw]{hprod_0_1_2-split.svg.pdf}
  \cbxfig<4\h0>[0.75\tw]{hprod_0_1_2_3-split.svg.pdf}
  \cbxfig<5\h1>[0.75\tw]{hprod_0_1_2_3_4-split.svg.pdf}

  \begin{onlyenv}<5\h1>
  \vspace{-2.7in}\hspace*{2.3in}
  \begin{minipage}{0.5\tw}
  \begin{cbxlist}

    \1 Choose language at the application, architecture, and VLSI level
       to emphasize productivity over performance

    \1 Possibly use a single language \\ at all abstraction levels

  \end{cbxlist}
  \end{minipage}
  \end{onlyenv}

\end{frame}

\begin{frame}{Previous Vertically Integrated Methodology}
  \cbxfigc<1\h0>{toolflow_0-split.svg.pdf}
  \cbxfigc<2\h0>{toolflow_0_1-split.svg.pdf}
  \cbxfigc<3\h0>{toolflow_0_1_2-split.svg.pdf}
  \cbxfigc<4\h0>{toolflow_0_1_2_3-split.svg.pdf}
  \cbxfigc<5-\h1>{toolflow_0_1_2_3_4-split.svg.pdf}
\end{frame}

\begin{frame}{Python-Based Vertically Integrated Methodology}
  \cbxfigc<1\h0>{toolflow_0_1_2_3_4-split.svg.pdf}
  \cbxfigc<2\h0>{toolflow_0_1_2_3_4_5-split.svg.pdf}
  \cbxfigc<3\h0>{toolflow_0_1_2_3_4_5_6-split.svg.pdf}
\end{frame}

\begin{frame}{Why Python?}

  \cbxfloatright{\cbxfigc[0.21\tw]{python-logo.svg.pdf}}

  \begin{cbxlist}

    \1 Python is well regarded as a highly productive \\ language with
       lightweight, pseudocode-like syntax

    \1 Python supports modern language features to \\ enable rapid, agile
       development (dynamic typing, \\ reflection, metaprogramming)

    \1 Python has a large and active developer and support community

    \1 Python includes extensive standard and third-party libraries

    \1 Python enables embedded domain-specific languages

    \1 Python facilitates engaging application-level researchers

    \1 Python includes built-in support for integrating with C/C++

    \1 Python performance is improving with advanced JIT compilation

  \end{cbxlist}

\end{frame}

\begin{frame}{What is PyMTL?}
  \insertslides{pymtl-intro}{9}{13}
\end{frame}

\begin{frame}{What is PyMTL for and not (currently) for?}
\begin{cbxlist}

  \1 \BF{PyMTL is for ...}

     \2 Taking an accelerator design from concept to implementation
     \2 Construction of highly-parameterizable RTL chip generators
     \2 Rapid design, testing, and exploration of hardware mechanisms
     \2 Quickly prototyping models and interfacing them with GEM5
     \2 Interfacing models with imported Verilog

  \1 \BF{PyMTL is not (currently) for ...}

     \2 Python high-level synthesis
     \2 Many-core simulations with hundreds of cores
     \2 Full-system simulation with real OS support
     \2 Users needing a complex OOO processor model ``out of the box''
     \2 Users needing an ARM/x86 processor model ``out of the box''
     \2 Users needing a mature modeling framework that will not change

\end{cbxlist}
\end{frame}

\begin{frame}{What is Pydgin?}

  \begin{center}
    Pydgin is a Python-based framework for productively \\ generating
    very fast instruction-set simulators
  \end{center}

  \cbxfigc{what-is-pydgin.pdf}

  \medskip\centering
  \begin{cbxlist}[t]

    \1 Flexible, productive, pseudocode-like ADL syntax
    \1 ADL embedded in a popular, general-purpose language
    \1 Tracing-JIT generator applies across many different ISAs
    \1 Leverages advancements from dynamic-language JIT research
    \1 Capable of simulating RISC instruction sets at 100's of MIPS

  \end{cbxlist}

\end{frame}

\begin{frame}{What is Pydgin for and not (currently) for?}
\begin{cbxlist}

  \1 \BF{Pydgin is for ...}

     \2 Building your own very fast instruction set simulators
     \2 Experimenting with emerging research instruction sets
     \2 Easily instrumenting an instruction set simulator for early analysis

  \1 \BF{Pydgin is not (currently) for ...}

     \2 Multi-core simulations (planned for the near future)
     \2 Full-system simulation
     \2 Users needing  an ARMv8/x86 simulator ``out of the box''
     \2 Users needing a mature modeling framework that will not change

\end{cbxlist}
\end{frame}

\begin{frame}

\vspace{0.5in}
\begin{cbxcols}

  \begin{column}{0.48\tw}
    \centering

    \cbxfigc{pymtl-flat.svg.pdf}
    \vspace{0.215in}

    PyMTL: A Unified Framework for Vertically Integrated Computer
    Architecture Research

    \vspace{0.39in}
    [ MICRO 2014 ]

    \vspace{0.05in}
    \small{https://github.com/cornell-brg/pymtl}
  \end{column}

  \begin{column}{0.48\tw}
    \centering

    \cbxfigc{pydgin-flat.svg.pdf}

    \vspace{0.25in}

    Pydgin: Generating Fast Instruction Set Simulators from Simple
    Architecture Descriptions with Meta-Tracing JIT Compilers

    \vspace{0.2in}
    [ ISPASS 2015 ]

    \vspace{0.05in}
    \small{https://github.com/cornell-brg/pydgin}

  \end{column}
\end{cbxcols}
\end{frame}

\begin{frame}{Tutorial Organizers}
\begin{cbxcols}

  \begin{column}{0.25\tw}

    \cbxfigbc{derek-lockhart.jpg}

    \medskip
    \cbxfigbc{berkin-ilbeyi.jpg}

  \end{column}

  \begin{column}{0.7\tw}
  \cbxlistfontsize{\footnotesize}{\footnotesize}

  \BF{Derek Lockhart}

  \vspace{0.1in}\hspace*{0.5em}%
  \begin{cbxlist}[t]

    \1 N\textsuperscript{th}-Year Ph.D.~Candidate, ECE, Cornell University

    \vspace{0.02in}
    \1 Graduating this summer and heading to Google Platforms

    \vspace{0.02in}
    \1 Research Interests: Hardware design methodologies, computer
       architecture, VLSI design

    \vspace{0.02in}
    \1 Lead researcher/developer for PyMTL framework

    \vspace{0.02in}
    \1 Co-Lead researcher/developer for Pydgin framework

  \end{cbxlist}

  \medskip
  \BF{Berkin Ilbeyi}

  \vspace{0.1in}\hspace*{0.5em}%
  \begin{cbxlist}[t]

    \1 3\textsuperscript{nd}-Year Ph.D.~Candidate, ECE, Cornell University

    \vspace{0.02in}
    \1 Research Interests: Computer architecture, just-in-time
          compilation, novel hardware/software interfaces

    \vspace{0.02in}
    \1 First ``real'' user of PyMTL framework for XLOOPS project

    \vspace{0.02in}
    \1 Co-Lead researcher/developer for Pydgin framework

  \end{cbxlist}

  \cbxlistfontsizereset{}
  \end{column}

\end{cbxcols}
\end{frame}

\scheduleslide{0}


%=========================================================================
% Presentation: Pydgin Intro
%=========================================================================

\section[{\it Presentation} Pydgin Intro]{}
\scheduleslide{3}

\begin{frame}{The Pydgin Framework}
  \insertslides{pydgin-intro}{1}{2}
\end{frame}

\begin{frame}{The Pydgin ADL: ARMv5 Architectural State}
  \insertslide{pydgin-intro}{3}
\end{frame}

\begin{frame}{The Pydgin ADL: ARMv5 Encodings}
  \insertslide{pydgin-intro}{4}
\end{frame}

\begin{frame}{The Pydgin ADL: ARMv5 Instruction Semantics}
  \insertslides{pydgin-intro}{5}{6}
\end{frame}

\begin{frame}{An RPython Instruction Set Simulator}
  \insertslides{pydgin-intro}{7}{9}
\end{frame}

\begin{frame}{The RPython Translation Toolchain}
  \insertslides{pydgin-intro}{10}{11}
\end{frame}

\begin{frame}{An RPython ISS with JIT Annotations}
  \insertslides{pydgin-intro}{12}{13}
\end{frame}

\begin{frame}{The RPython Translation Toolchain JIT Generator}
  \insertslides{pydgin-intro}{14}{15}
\end{frame}

\begin{frame}{JIT Annotations}
  \insertslides{pydgin-intro}{16}{18}
\end{frame}

\begin{frame}{Pydgin ISS Performance}
  \insertslides{pydgin-intro}{19}{27}
\end{frame}

%=========================================================================
% Hands-On: GCD Instr
%=========================================================================

\stepcounter{taskseccount}
\section[{\it Hands-On} GCD Instr]{}
\scheduleslide{4}

%-------------------------------------------------------------------------
% GCD
%-------------------------------------------------------------------------

\begin{task}
\begin{frame}[fragile]{Write Euclid's greatest common divisor (GCD)
algorithm in Python interpreter}

\vspace{-20pt}

\begin{lstlisting}[numbers=none,basicstyle=\ttfamily]
% python
>>> def gcd( a, b ):
...   while b:
...     a, b = b, a%b
...   return a
...
>>> gcd( 1, 5 )
1
>>> gcd( 9, 3 )
3
>>> gcd( 9, 6 )
3
\end{lstlisting}

\end{frame}
\end{task}

%-------------------------------------------------------------------------
% PARC overview
%-------------------------------------------------------------------------

\begin{frame}[fragile]{PARC Overview}

\begin{itemize}
  \item MIPS-like 32-bit RISC ISA
  \item 32 general-purpose registers
  \item "Simple" instruction encodings
  \item Example instruction: add unsigned
\end{itemize}

\begin{verbatim}
  addu rd, rs, rt             R[rd] = R[rs] + R[rt]

  31    26 25   21 20   16 15   11 10    6 5      0
 +--------+-------+-------+-------+-------+--------+
 |   op   |  rs   |  rt   |  rd   |       |  cmd   |
 | 000000 | src0  | src1  | dst   | 00000 | 100001 |
 +--------+-------+-------+-------+-------+--------+

\end{verbatim}

\end{frame}

%-------------------------------------------------------------------------
% Pydgin framework overview
%-------------------------------------------------------------------------

\begin{frame}{Pydgin Framework Overview}

\begin{itemize}
  \item \texttt{pydgin/}
  \begin{itemize}
    \item Common, ISA-independent components
    \item Fetch-decode-execute loop, bitwise operations, register file,
    memory, syscall implementations, JIT annotations
    \item A user shouldn't need to modify these
  \end{itemize}
  \item \texttt{parc/}, \texttt{arm/}, \texttt{<your favorite isa>/}
  \begin{itemize}
    \item Architecture implementation
    \item \texttt{machine.py}: architectural state
    \item \texttt{instruction.py}: static instruction fields
    \item \texttt{isa.py}: instruction encodings and semantics
    \item \texttt{parc-sim.py}: executable
  \end{itemize}
\end{itemize}

\end{frame}

%-------------------------------------------------------------------------
% parc/machine.py
%-------------------------------------------------------------------------

\begin{frame}[fragile]{\texttt{parc/machine.py}}

\begin{lstlisting}[numbers=none,basicstyle=\ttfamily]
# the architectural state (simplified)
class State():
  def __init__( self, ... ):
    self.pc  = ...
    self.rf  = ...
    self.mem = ...

    # statistics
    self.num_insts      = ...
    self.stat_num_insts = ...
\end{lstlisting}

\end{frame}

%-------------------------------------------------------------------------
% parc/instruction.py
%-------------------------------------------------------------------------

\begin{frame}[fragile]{\texttt{parc/instruction.py}}

\begin{Verbatim}[commandchars=\\\{\}]
  addu \textcolor{red}{rd}, \textcolor{red}{rs}, \textcolor{red}{rt}             R[\textcolor{red}{rd}] = R[\textcolor{red}{rs}] + R[\textcolor{red}{rt}]

  31    26 25   21 20   16 15   11 10    6 5      0
 +--------+-------+-------+-------+-------+--------+
 |   op   |  \textcolor{red}{rs}   |  \textcolor{red}{rt}   |  \textcolor{red}{rd}   |       |  cmd   |
 | 000000 | src0  | src1  | dst   | 00000 | 100001 |
 +--------+-------+-------+-------+-------+--------+
\end{Verbatim}

\vspace{-20pt}

\begin{lstlisting}[numbers=none,basicstyle=\ttfamily]
class Instruction():
  # ...
  @property
  def rd( self ):
    return (self.bits >> 11) & 0x1F
\end{lstlisting}
\end{frame}

%-------------------------------------------------------------------------
% parc/isa.py
%-------------------------------------------------------------------------

\begin{frame}[fragile]{\texttt{parc/isa.py}}

\begin{Verbatim}[commandchars=\\\{\}]
  addu rd, rs, rt             R[rd] = R[rs] + R[rt]

  31    26 25   21 20   16 15   11 10    6 5      0
 +--------+-------+-------+-------+-------+--------+
 |   op   |  rs   |  rt   |  rd   |       |  cmd   |
 | \textcolor{red}{000000} | src0  | src1  | dst   | \textcolor{red}{00000} | \textcolor{red}{100001} |
 +--------+-------+-------+-------+-------+--------+
\end{Verbatim}

\vspace{-20pt}

\begin{lstlisting}[numbers=none,basicstyle=\ttfamily]
# instruction encodings
encodings = [
  # ...
  ['addu', '000000_xxxxx_xxxxx_xxxxx_00000_100001'],
  # ...
]
\end{lstlisting}
\end{frame}

%-------------------------------------------------------------------------
% parc/isa.py (continued)
%-------------------------------------------------------------------------

\begin{frame}[fragile]{\texttt{parc/isa.py} (continued)}

\begin{Verbatim}[commandchars=\\\{\}]
  addu rd, rs, rt             \textcolor{red}{R[rd] = R[rs] + R[rt]}

  31    26 25   21 20   16 15   11 10    6 5      0
 +--------+-------+-------+-------+-------+--------+
 |   op   |  rs   |  rt   |  rd   |       |  cmd   |
 | 000000 | src0  | src1  | dst   | 00000 | 100001 |
 +--------+-------+-------+-------+-------+--------+

\end{Verbatim}

\vspace{-20pt}

\begin{lstlisting}[numbers=none]
# addu semantics
# s    = state
# inst = instruction bits
def execute_addu( s, inst ):
  s.rf[ inst.rd ] = trim_32( s.rf[ inst.rs ] + s.rf[ inst.rt ] )
  s.pc += 4
\end{lstlisting}
\end{frame}

%-------------------------------------------------------------------------
% Tasklist
%-------------------------------------------------------------------------

\begin{frame}{\IT{Hands-On:} Add and evaluate GCD inst in Pydgin}
\begin{cbxlist}
  \1 Given: Extend the cross-compiler
  \1 Given: Add assembly test
  \1 Task 1.1: Build and run assembly tests using cross-compiler
  \1 Task 1.2: Investigate test failures
  \1 Task 1.3: Add encoding for \texttt{gcd} instruction
  \1 Task 1.4: Implement semantics
  \1 Task 1.5: Re-run assembly tests
  \1 Task 1.6: Count the GCD cycles
  \1 Task 1.7: Add inline assembly to application
  \1 Task 1.8: Translate and evaluate
\end{cbxlist}
\end{frame}

\begin{frame}{\IT{Hands-On:} Add and evaluate GCD inst in Pydgin}
\begin{cbxlist}
  \1 \BF{Given: Extend the cross-compiler}
  \1 \BF{Given: Add assembly test}
  \1 Task 1.1: Build and run assembly tests using cross-compiler
  \1 Task 1.2: Investigate test failures
  \1 Task 1.3: Add encoding for \texttt{gcd} instruction
  \1 Task 1.4: Implement semantics
  \1 Task 1.5: Re-run assembly tests
  \1 Task 1.6: Count the GCD cycles
  \1 Task 1.7: Add inline assembly to application
  \1 Task 1.8: Translate and evaluate
\end{cbxlist}
\end{frame}

\begin{frame}{\IT{Hands-On:} Add and evaluate GCD inst in Pydgin}
\begin{cbxlist}
  \1 Given: Extend the cross-compiler
  \1 Given: Add assembly test
  \1 \BF{Task 1.1: Build and run assembly tests using cross-compiler}
  \1 \BF{Task 1.2: Investigate test failures}
  \1 Task 1.3: Add encoding for \texttt{gcd} instruction
  \1 Task 1.4: Implement semantics
  \1 Task 1.5: Re-run assembly tests
  \1 Task 1.6: Count the GCD cycles
  \1 Task 1.7: Add inline assembly to application
  \1 Task 1.8: Translate and evaluate
\end{cbxlist}
\end{frame}

%-------------------------------------------------------------------------
% Build and run asm tests
%-------------------------------------------------------------------------

\begin{task}
\begin{frame}[fragile]{Build and run assembly tests using cross-compiler}

\begin{Verbatim}[commandchars=\\\{\}]
% cd \midtilde/pydgin-tut/parc/asm_tests_build
% make
\end{Verbatim}

The following should pass except for gcd:

\begin{verbatim}
% make check
\end{verbatim}

{\small
\begin{verbatim}
parc-gcd.out:Exception in execution (pc: 0x00808008), aborting!
parc-gcd.out:Exception message: Invalid instruction 0x9c411811!
\end{verbatim}}

\end{frame}
\end{task}

%-------------------------------------------------------------------------
% Investigate test failures
%-------------------------------------------------------------------------

\begin{task}
\begin{frame}[fragile]{Investigate test failures}

{\small
\begin{verbatim}
parc-gcd.out:Exception in execution (pc: 0x00808008), aborting!
parc-gcd.out:Exception message: Invalid instruction 0x9c411811!
\end{verbatim}}

\begin{Verbatim}[commandchars=\\\{\}]
% cd \midtilde/pydgin-tut/parc/asm_tests/build
% maven-objdump -dC parc-gcd > gcd.dump
% less gcd.dump
\end{Verbatim}

{\small
\begin{verbatim}
  808000:       24010018        li      at,24
  808004:       24020064        li      v0,100
  808008:       9c411811        gcd     v1,v0,at
  80800c:       24040004        li      a0,4
  808010:       241d000e        li      sp,14
  808014:       14640032        bne     v1,a0,8080e0 <_fail>
\end{verbatim}}

\begin{Verbatim}[commandchars=\\\{\}]
% less \midtilde/pydgin-tut/parc/asm_tests/parc/parc-gcd.S
\end{Verbatim}

\end{frame}
\end{task}

%-------------------------------------------------------------------------
% Tasklist
%-------------------------------------------------------------------------

\begin{frame}{\IT{Hands-On:} Add and evaluate GCD inst in Pydgin}
\begin{cbxlist}
  \1 Given: Extend the cross-compiler
  \1 Given: Add assembly test
  \1 Task 1.1: Build and run assembly tests using cross-compiler
  \1 Task 1.2: Investigate test failures
  \1 \BF{Task 1.3: Add encoding for \texttt{gcd} instruction}
  \1 \BF{Task 1.4: Implement semantics}
  \1 \BF{Task 1.5: Re-run assembly tests}
  \1 Task 1.6: Count the GCD cycles
  \1 Task 1.7: Add inline assembly to application
  \1 Task 1.8: Translate and evaluate
\end{cbxlist}
\end{frame}

%-------------------------------------------------------------------------
% Add gcd encoding
%-------------------------------------------------------------------------

\begin{task}
\begin{frame}[fragile]{Add encoding for \texttt{gcd} instruction}

\begin{Verbatim}[commandchars=\\\{\}]
  gcd rd, rs, rt        R[rd] = gcd( R[rs], R[rt] )

  31    26 25   21 20   16 15   11 10    6 5      0
 +--------+-------+-------+-------+-------+--------+
 |   op   |  rs   |  rt   |  rd   |       |  cmd   |
 | \textcolor{red}{100111} | src0  | src1  | dst   | \textcolor{red}{xxxxx} | \textcolor{red}{010001} |
 +--------+-------+-------+-------+-------+--------+
\end{Verbatim}

\vspace{-10pt}

\begin{Verbatim}[commandchars=\\\{\}]
% cd \midtilde/pydgin-tut/parc
% gedit isa.py
\end{Verbatim}

\vspace{-10pt}

\begin{lstlisting}[firstnumber=243,basicstyle=\ttfamily]
# TASK: pick the correct encoding
['gcd', '111111_11111_11111_11111_11111_111111'],
\end{lstlisting}
\end{frame}
\end{task}

%-------------------------------------------------------------------------
% Implement semantics
%-------------------------------------------------------------------------

\begin{task}
\begin{frame}[fragile]{Implement semantics}

\vspace{-15pt}

\begin{Verbatim}[commandchars=\\\{\}]
% cd \midtilde/pydgin-tut/parc
% gedit isa.py
\end{Verbatim}

\begin{lstlisting}[firstnumber=319]
def execute_addu( s, inst ):
  s.rf[ inst.rd ] = trim_32( s.rf[ inst.rs ] + s.rf[ inst.rt ] )
  s.pc += 4
\end{lstlisting}
\vspace{-10pt}
\begin{lstlisting}[firstnumber=976]
def execute_gcd( s, inst ):
  # TASK: implement gcd here
  s.pc += 4
\end{lstlisting}
\vspace{-10pt}
\begin{lstlisting}[numbers=none]
def gcd( a, b ):
  while b:
    a, b = b, a%b
  return a
\end{lstlisting}

\end{frame}
\end{task}

%-------------------------------------------------------------------------
% Re-run asm tests
%-------------------------------------------------------------------------

\begin{task}
\begin{frame}[fragile]{Re-run assembly tests}

\begin{Verbatim}[commandchars=\\\{\}]
% cd \midtilde/pydgin-tut/parc/asm_tests/build

% parc-sim.py --test parc-gcd
\end{Verbatim}

Try different debug flags:

{\small
\begin{verbatim}
% parc-sim.py --test --debug insts         parc-gcd | less
% parc-sim.py --test --debug insts,rf,mem  parc-gcd | less
% parc-sim.py --test --debug insts,regdump parc-gcd | less
\end{verbatim}}
{\tiny
\begin{verbatim}
808000 24010018 addiu    0        :: RD.RF[0 ] = 00000000 :: WR.RF[1 ] = 00000018
808004 24020064 addiu    1        :: RD.RF[0 ] = 00000000 :: WR.RF[2 ] = 00000064
808008 9c411811 gcd      2        :: RD.RF[2 ] = 00000064 :: RD.RF[1 ] = 00000018 :: WR.RF[3 ] = 00000004
80800c 24040004 addiu    3        :: RD.RF[0 ] = 00000000 :: WR.RF[4 ] = 00000004
808010 241d000e addiu    4        :: RD.RF[0 ] = 00000000 :: WR.RF[29] = 0000000e
808014 14640032 bne      5        :: RD.RF[3 ] = 00000004 :: RD.RF[4 ] = 00000004
\end{verbatim}}
\end{frame}
\end{task}

%-------------------------------------------------------------------------
% Tasklist
%-------------------------------------------------------------------------

\begin{frame}{\IT{Hands-On:} Add and evaluate GCD inst in Pydgin}
\begin{cbxlist}
  \1 Given: Extend the cross-compiler
  \1 Given: Add assembly test
  \1 Task 1.1: Build and run assembly tests using cross-compiler
  \1 Task 1.2: Investigate test failures
  \1 Task 1.3: Add encoding for \texttt{gcd} instruction
  \1 Task 1.4: Implement semantics
  \1 Task 1.5: Re-run assembly tests
  \1 \BF{Task 1.6: Count the GCD cycles}
  \1 \BF{Task 1.7: Add inline assembly to application}
  \1 \BF{Task 1.8: Translate and evaluate}
\end{cbxlist}
\end{frame}

%-------------------------------------------------------------------------
% Count GCD cycles
%-------------------------------------------------------------------------

\begin{task}
\begin{frame}[fragile]{Count the GCD cycles}

{}How many times do we loop when executing the \texttt{gcd} instruction?

\begin{itemize}
  \item gcd semantics (\texttt{parc/isa.py})
  \item architectural state (\texttt{parc/machine.py})
  \item uncomment to print the counter (\texttt{parc/parc-sim.py})
\end{itemize}

\begin{Verbatim}[commandchars=\\\{\}]
% cd \midtilde/pydgin-tut/parc/asm_tests/build
% parc-sim.py --test parc-gcd
  [ passed ] parc-gcd
DONE! Status = 0
Instructions Executed = 15
Instructions Executed in Stat Region = 0
Number of cycles in GCD = 18
Total number of cycles (assuming non-GCD CPI=1) = 33
\end{Verbatim}

\end{frame}
\end{task}

%-------------------------------------------------------------------------
% Add inline asm to app
%-------------------------------------------------------------------------

\begin{task}
\begin{frame}[fragile]{Add inline assembly to application}

\vspace{-10pt}

% TODO: this slide is too long as well
\begin{Verbatim}[commandchars=\\\{\}]
% cd \midtilde/pydgin-tut/bmarks/ubmark
% gedit ubmark-gcd.cc
\end{Verbatim}

\vspace{-20pt}

\begin{lstlisting}[language=c,firstnumber=66]
int gcd_hw( int a, int b ) {
  int result;
  __asm__( "gcd %0, %1, %2" : "=r" (result)
                            : "r" (a), "r" (b) );
  return result;
}
\end{lstlisting}
\vspace{-30pt}
\begin{lstlisting}[language=c,firstnumber=76]
void gcd_scalar_accel( ... ) {
  for ( int i = 0; i < size; i++ )
    dest[i] = gcd_hw( src0[i], src1[i] );
}
\end{lstlisting}

\vspace{-20pt}

\begin{Verbatim}[commandchars=\\\{\}]
% cd \midtilde/pydgin-tut/bmarks/build
% make ubmark-gcd
% parc-sim.py ubmark-gcd --impl all --verify
\end{Verbatim}

\end{frame}
\end{task}

%-------------------------------------------------------------------------
% Translate and evaluate
%-------------------------------------------------------------------------

\begin{task}
\begin{frame}[fragile]{Translate and evaluate}

\begin{Verbatim}[commandchars=\\\{\}]
% cd \midtilde/pydgin-tut/scripts
% ./build.py pydgin-parc-nojit-debug
\end{Verbatim}

Evaluate using interpretive and translated Pydgin

\begin{Verbatim}[commandchars=\\\{\}]
% cd \midtilde/pydgin-tut/bmarks/build
% parc-sim.py             ubmark-gcd --impl scalar
% parc-sim.py             ubmark-gcd --impl scalar-accel
% pydgin-parc-nojit-debug ubmark-gcd --impl scalar
% pydgin-parc-nojit-debug ubmark-gcd --impl scalar-accel
\end{Verbatim}

If you've implemented everything correctly, you should get 163,513 and
108,414 cycles for \texttt{scalar} and \texttt{scalar-accel} respectively.

\end{frame}
\end{task}


%=========================================================================
% Presentation: PyMTL Intro
%=========================================================================

\section[Presentation: PyMTL Intro]{Presentation: PyMTL Introduction}

\begin{frame}{Computer Architecture Research Methodologies}
  \cbxfigc<1\h0>{pymtl-stack_0-split.svg.pdf}
  \cbxfigc<2\h0>{pymtl-stack_0_1_1a-split.svg.pdf}
  \cbxfigc<3\h0>{pymtl-stack_0_1_2_2a-split.svg.pdf}
  \cbxfigc<4\h0>{pymtl-stack_0_1_2_3_3a-split.svg.pdf}
  \cbxfigc<5\h1>{pymtl-stack_0_1_2_3_4-split.svg.pdf}
  \cbxfigc<6\h0>{pymtl-stack_0_4_5-split.svg.pdf}
  \cbxfigc<7\h2>{pymtl-stack_4_5_6-split.svg.pdf}
\end{frame}

\begin{frame}{Great Ideas From Prior Frameworks}
  \insertslide{pymtl-intro}{8}
\end{frame}

\begin{frame}{What is PyMTL?}
  \insertslides{pymtl-intro}{9}{13}
\end{frame}

\begin{frame}{The PyMTL Framework}
  \insertslides{pymtl-intro}{14}{19}
\end{frame}

\begin{frame}{What is PyMTL for and not (currently) for?}
\begin{cbxlist}

  \1 \BF{PyMTL is for ...}

     \2 Taking an accelerator design from concept to implementation
     \2 Construction of highly-parameterizable RTL chip generators
     \2 Rapid design, testing, and exploration of hardware mechanisms
     \2 Quickly prototyping models and interfacing them with GEM5
     \2 Interfacing models with imported Verilog

  \1 \BF{PyMTL is not (currently) for ...}

     \2 Python high-level synthesis
     \2 Many-core simulations with hundreds of cores
     \2 Full-system simulation with real OS support
     \2 Users needing a complex OOO processor model ``out of the box''
     \2 Users needing an ARM/x86 processor model ``out of the box''
     \2 Users needing a mature modeling framework that will not change

\end{cbxlist}
\end{frame}

\begin{frame}{Why Python?}

  \cbxfloatright{\cbxfigc[0.21\tw]{python-logo.svg.pdf}}

  \begin{cbxlist}

    \1 Python is well regarded as a highly productive \\ language with
       lightweight, pseudocode-like syntax

    \1 Python supports modern language features to \\ enable rapid, agile
       development (dynamic typing, \\ reflection, metaprogramming)

    \1 Python has a large and active developer and support community

    \1 Python includes extensive standard and third-party libraries

    \1 Python enables embedded domain-specific languages

    \1 Python facilitates engaging application-level researchers

    \1 Python includes built-in support for integrating with C/C++

    \1 Python performance is improving with advanced JIT compilation

  \end{cbxlist}

\end{frame}

\begin{frame}{PyMTL 101: Traditional Model in Python}
  \insertslide{pymtl-intro}{20}
\end{frame}

\begin{frame}{PyMTL 101: Model in PyMTL Embedded DSL}
  \insertslides{pymtl-intro}{21}{25}
\end{frame}


%=========================================================================
% Hands-On: RegIncr
%=========================================================================

\stepcounter{taskseccount}
\section[{\it Hands-On} Max/RegIncr]{}
\scheduleslide{6}

\begin{frame}{\IT{Hands-On:} PyMTL Basics with Max/RegIncr}
\begin{cbxlist}
  \1 Task 2.1: Experiment with \TT{Bits}
  \1 Task 2.2: Interactively simulate a max unit
  \1 Task 2.3: Write a registered incrementer (RegIncr) model
  \1 Task 2.4: Test the RegIncr model
  \1 Task 2.5: Translate the RegIncr model into Verilog
  \1 Task 2.6: Simulate the RegIncr model with line tracing
  \1 Task 2.7: Simulate the RegIncr model with VCD dumping
  \1 Task 2.8: Compose a pipeline with two RegIncr models
  \1 Task 2.9: Compose a pipeline with N RegIncr models
  \1 Task 2.10: Parameterize test to verify multiple Ns
\end{cbxlist}
\end{frame}

\begin{frame}{\IT{Hands-On:} PyMTL Basics with Max/RegIncr}
\begin{cbxlist}
  \1 \BF{Task 2.1: Experiment with \TT{Bits}}
  \1 \BF{Task 2.2: Interactively simulate a max unit}
  \1 Task 2.3: Write a registered incrementer (RegIncr) model
  \1 Task 2.4: Test the RegIncr model
  \1 Task 2.5: Translate the RegIncr model into Verilog
  \1 Task 2.6: Simulate the RegIncr model with line tracing
  \1 Task 2.7: Simulate the RegIncr model with VCD dumping
  \1 Task 2.8: Compose a pipeline with two RegIncr models
  \1 Task 2.9: Compose a pipeline with N RegIncr models
  \1 Task 2.10: Parameterize test to verify multiple Ns
\end{cbxlist}
\end{frame}

%-------------------------------------------------------------------------
\begin{frame}[fragile]{\TT{Bits} Class for Fixed-Bitwidth Values}
%-------------------------------------------------------------------------
\colorbox{gray!30!white}{\parbox{\tw}{\rule[-0.4em]{0pt}{1.4em}\centering\textbf{%
  PyMTL Bits Operators
}}}

{\scriptsize

\newenvironment{optbl2}[1]
{
  %\vspace{0.5em}
  \centering{\BF{\footnotesize{{#1}}}}
  \smallskip

  %\hspace{0.75em}
  \begin{tabular}{>{\ttfamily\centering\arraybackslash}p{0.23\tw}p{.76\tw}}
}{
  \end{tabular}
}

\newenvironment{optbl}[1]
{
  %\vspace{0.5em}
  \centering{\BF{\footnotesize{{#1}}}}
  \smallskip

  %\hspace{0.75em}
  \begin{tabular}{>{\ttfamily\centering\arraybackslash}p{0.33\tw}p{.66\tw}}
%\toprule
}{
  \end{tabular}
}

\begin{cbxcols}
\begin{column}{0.30\tw}
\vspace{.3in}

\begin{optbl2}{Logical Operators}
  \verb|&|   & bitwise AND   \\
  |          & bitwise OR    \\
  \verb|^|   & bitwise XOR   \\
  \verb|^~|  & bitwise XNOR  \\
  \verb|~|   & bitwise NOT   \\
\end{optbl2}

\vspace{.15in}
\begin{optbl2}{Arith. Operators}
  \verb|+|   & addition         \\
  \verb|-|   & subtraction      \\
  \verb|*|   & multiplication   \\
  \verb|/|   & division         \\
  \verb|%|   & modulo           \\
\end{optbl2}


\end{column}
\begin{column}{0.30\tw}
\vspace{.3in}

\begin{optbl}{Shift Operators}
  \verb|>>|  & shift right            \\
  \verb|<<|  & shift left             \\
\end{optbl}

\vspace{.15in}

\begin{optbl}{Slice Operators}
  \verb|[x]|   & get/set bit x            \\
  \verb|[x:y]| & get/set bits             \\
               &  x upto y                \\

\end{optbl}

\vspace{.15in}

\begin{optbl}{Reduction Operators}
  \verb|reduce_and| & reduce via AND \\
  \verb|reduce_or|  & reduce via OR  \\
  \verb|reduce_xor| & reduce via XOR \\
\end{optbl}

\end{column}
\begin{column}{0.40\tw}
\vspace{.3in}

\begin{optbl2}{Relational Operators}
  \verb|==|  & equal                  \\
  \verb|!=|  & not equal              \\
  \verb|>|   & greater than           \\
  \verb|>=|  & greater than or equals \\
  \verb|<|   & less than              \\
  \verb|<=|  & less than or equals    \\
\end{optbl2}

\vspace{.15in}

\begin{optbl2}{Other Functions}
  \verb|concat| & concatenate         \\
  \verb|sext|   & sign-extension      \\
  \verb|zext|   & zero-extension      \\
\end{optbl2}

\end{column}
\end{cbxcols}

}

\end{frame}

%-------------------------------------------------------------------------
\begin{task}\begin{frame}[fragile]{Experiment with \TT{Bits}}
%-------------------------------------------------------------------------

\vspace{-0.15in}
\begin{cbxcols}

\begin{column}{0.4\tw}
\begin{lstlisting}[xleftmargin={0in},numbers={none},basicstyle={\ttfamily\small}]
 % cd ~
 % ipython

 >>> from pymtl import *
 >>> a = Bits( 8, 5 )
 >>> b = Bits( 8, 3 )
 >>> a + b
 Bits( 8, 0x08 )
 >>> a - b
 Bits( 8, 0x02 )
 >>> a | b
 Bits( 8, 0x07 )
 >>> a & b
 Bits( 8, 0x01 )
\end{lstlisting}
\end{column}

\begin{column}{0.4\tw}
\begin{lstlisting}[xleftmargin={0in},numbers={none},basicstyle={\ttfamily\small}]



 >>> c = concat( a, b )
 >>> c
 Bits( 16, 0x0503 )
 >>> c[0:8]
 Bits( 8, 0x03 )
 >>> c[8:16]
 Bits( 8, 0x05 )
 >>> exit()
\end{lstlisting}
\end{column}

\end{cbxcols}
\end{frame}
\end{task}

%-------------------------------------------------------------------------
\begin{task}\begin{frame}[fragile]{Interactively simulate a max unit}
%-------------------------------------------------------------------------

\vspace{-0.15in}
\begin{lstlisting}[xleftmargin={0in},numbers={none},basicstyle={\ttfamily\small}]
 % cd ~/pymtl-tut/maxunit
 % ipython

 >>> from pymtl import *
 >>> from MaxUnitFL import MaxUnitFL
 >>> model = MaxUnitFL( nbits=8, nports=3 )
 >>> model.elaborate()
 >>> sim = SimulationTool(model)
 >>> sim.reset()
 >>> model.in_[0].value = 2
 >>> model.in_[1].value = 5
 >>> model.in_[2].value = 3
 >>> sim.cycle()
 >>> model.out
 Bits( 8, 0x05 )
 >>> exit()
\end{lstlisting}

\end{frame}
\end{task}

%-------------------------------------------------------------------------
%\begin{task}\begin{frame}[fragile]{Run RegIncrFL and RegIncrRTL tests}
%-------------------------------------------------------------------------
% \begin{verbatim}
%   % mkdir ~/pymtl-tut/build
%   % cd      ~/pymtl-tut/build
% \end{verbatim}
%
% \begin{verbatim}
%   % py.test ../regincr/RegIncrFL_test.py --verbose
%   % py.test ../regincr/RegIncrRTL_test.py --verbose
% \end{verbatim}
%
% \begin{centering}
% \TT{RegIncrFL\_test.py} should pass. \\*
% \TT{RegIncrRTL\_test.py} should \BF{fail}!
%
% \vspace{0.4in}
% We haven't implemented it yet!
%
% \end{centering}
%
% \end{frame}
% \end{task}

\begin{frame}{\IT{Hands-On:} PyMTL Basics with Max/RegIncr}
\begin{cbxlist}
  \1 Task 2.1: Experiment with \TT{Bits}
  \1 Task 2.2: Interactively simulate a max unit
  \1 \BF{Task 2.3: Write a registered incrementer (RegIncr) model}
  \1 \BF{Task 2.4: Test the RegIncr model}
  \1 \BF{Task 2.5: Translate the RegIncr model into Verilog}
  \1 Task 2.6: Simulate the RegIncr model with line tracing
  \1 Task 2.7: Simulate the RegIncr model with VCD dumping
  \1 Task 2.8: Compose a pipeline with two RegIncr models
  \1 Task 2.9: Compose a pipeline with N RegIncr models
  \1 Task 2.10: Parameterize test to verify multiple Ns
\end{cbxlist}
\end{frame}

%-------------------------------------------------------------------------
\begin{task}\begin{frame}[fragile]{Write a registered incrementer model}
%-------------------------------------------------------------------------

\vspace{-0.15in}
\begin{Verbatim}[commandchars=\\\{\}]
 % cd    \midtilde/pymtl-tut/build
 % gedit ../regincr/RegIncrRTL.py
\end{Verbatim}
\vspace{-0.17in}

\lstinputlisting[xleftmargin={0.38in},firstline=8,lastline=21,firstnumber=8]%
{../../regincr_soln/RegIncrRTL.py}

\vspace{-0.2in}
\begin{Verbatim}
 % py.test ../regincr/RegIncrRTL_test.py --verbose
\end{Verbatim}
\end{frame}
\end{task}

%-------------------------------------------------------------------------
\begin{task}\begin{frame}[fragile]{Test the RegIncr model}
%-------------------------------------------------------------------------

\vspace{-0.19in}
\begin{Verbatim}[commandchars=\\\{\}]
 % cd    \midtilde/pymtl-tut/build
 % py.test ../regincr/RegIncrRTL_test.py --verbose
\end{Verbatim}
\smallskip
\begin{Verbatim}
 =============== test session starts ================
 platform darwin -- Python 2.7.5 -- pytest-2.6.4
 plugins: xdist
 collected 2 items

 ../regincr/RegIncrRTL_test.py::test_simple[8] PASSED
 ../regincr/RegIncrRTL_test.py::test_random[8] PASSED

 ============= 2 passed in 0.36 seconds =============
\end{Verbatim}

\end{frame}
\end{task}

%-------------------------------------------------------------------------
\begin{frame}{Translating PyMTL RTL into Verilog}
%-------------------------------------------------------------------------

\begin{cbxlist}

  \1 PyMTL models written at the register-transfer level of abstraction
     can be translated into Verilog source using the \TT{TranslationTool}

  \1 Generated Verilog can be used with commercial EDA toolflows to
    characterize area, energy, and timing

\end{cbxlist}

\vspace{0.3in}
\cbxfigc{pymtl-tut-refine.pdf}

\end{frame}

%-------------------------------------------------------------------------
\begin{frame}{Translation as Part of PyMTL SimJIT-RTL}
%-------------------------------------------------------------------------

  \vspace{0.2in}
  \cbxfigc<1>{pymtl-tut-simjit0.pdf}
  \cbxfigc<2>{pymtl-tut-simjit1.pdf}

\end{frame}

%-------------------------------------------------------------------------
\begin{frame}{PyMTL to Verilog Translation Limitations}
%-------------------------------------------------------------------------

{}The \BF{TranslationTool} has limitations on what it can translate:

\medskip
\begin{cbxlist}[ll]

  \1 \BF{Static elaboration} can use arbitratry Python \\ (connections
        $\Rightarrow$ connectivity graph $\Rightarrow$ structural
        Verilog)

  \1 \BF{Concurrent logic blocks} must abide by language restrictions

  \pause

    \2 Data must be communicated in/out/between blocks using \BF{signals} \\
       (InPorts, OutPorts, and Wires)

    \2 Signals may only contain \BF{bit-specific value types} \\
       (Bits/BitStructs)

    \2 Only pre-defined, \BF{translatable operators/functions} may be
       used \\ (no user-defined operators or functions)

    \2 Any variables that don't refer to signals must be \BF{integer
       constants}

\end{cbxlist}
\end{frame}

%-------------------------------------------------------------------------
\begin{task}\begin{frame}[fragile]{Translate the RegIncr model into Verilog}
%-------------------------------------------------------------------------

\vspace{-0.15in}
\begin{Verbatim}[commandchars=\\\{\}]
 % cd    \midtilde/pymtl-tut/build
 % gedit ../regincr/RegIncrRTL_test.py
\end{Verbatim}
\vspace{-0.17in}

\lstinputlisting[xleftmargin={0.38in},firstline=20,lastline=29,firstnumber=20]%
{../../regincr_soln/RegIncrRTL_test.py}

\vspace{-0.2in}
\begin{Verbatim}[commandchars=\\\{\}]
 % py.test ../regincr/RegIncrRTL_test.py -v --test-verilog
 % gedit ./RegIncrRTL_0x216e72a4.v
\end{Verbatim}

\end{frame}
\end{task}

%-------------------------------------------------------------------------
\begin{frame}[fragile]{Example Verilog Generated from Translation}
%-------------------------------------------------------------------------

\vspace{-0.25in}
\begin{cbxcols}

\begin{column}{0.5\tw}
\lstinputlisting[xleftmargin={0.2in},firstline=4,lastline=25,firstnumber=4,basicstyle={\ttfamily\scriptsize}]%
{../../regincr_soln/RegIncrRTL_0x7a355c5a216e72a4.v}
\end{column}

\begin{column}{0.5\tw}
\lstinputlisting[xleftmargin={0.2in},firstline=26,lastline=50,firstnumber=26,basicstyle={\ttfamily\scriptsize}]%
{../../regincr_soln/RegIncrRTL_0x7a355c5a216e72a4.v}
\end{column}

\end{cbxcols}
\end{frame}

\begin{frame}{\IT{Hands-On:} PyMTL Basics with Max/RegIncr}
\begin{cbxlist}
  \1 Task 2.1: Experiment with \TT{Bits}
  \1 Task 2.2: Interactively simulate a max unit
  \1 Task 2.3: Write a registered incrementer (RegIncr) model
  \1 Task 2.4: Test the RegIncr model
  \1 Task 2.5: Translate the RegIncr model into Verilog
  \1 \BF{Task 2.6: Simulate the RegIncr model with line tracing}
  \1 \BF{Task 2.7: Simulate the RegIncr model with VCD dumping}
  \1 Task 2.8: Compose a pipeline with two RegIncr models
  \1 Task 2.9: Compose a pipeline with N RegIncr models
  \1 Task 2.10: Parameterize test to verify multiple Ns
\end{cbxlist}
\end{frame}

%-------------------------------------------------------------------------
\begin{frame}{Unit Tests vs. Simulators}
%-------------------------------------------------------------------------

\BF{Unit Tests:} \TT{ModelName\_tests.py}
\begin{itemize}
  \item Tests that verify the simulation behavior of a model isolation
  \item Test functions are executed by the \TT{py.test} testing framework
  \item Unit tests should always be written before simulator scripts!
\end{itemize}

\vspace{0.2in}

\BF{Simulators:} \TT{model-name-sim.py}
\begin{itemize}
  \item Simulators are meant for model evaluation and stats collection
  \item Simulation scripts take commandline arguments for configuration
  \item Used for experimentation and (design space) exploration!
\end{itemize}

\end{frame}

%-------------------------------------------------------------------------
\begin{task}\begin{frame}[fragile]{Simulate RegIncr with line tracing}
%-------------------------------------------------------------------------

\vspace{-0.15in}
\begin{Verbatim}[commandchars=\\\{\}]
 % cd \midtilde/pymtl-tut/build
 % python ../regincr/reg-incr-sim.py 10
 % python ../regincr/reg-incr-sim.py 10 --trace
 % python ../regincr/reg-incr-sim.py 20 --trace

   0: 04e5f14d (00000000) 00000000
   1: 7839d4fc (04e5f14d) 04e5f14e
   2: 996ab63d (7839d4fc) 7839d4fd
   3: 6d146dfc (996ab63d) 996ab63e
   4: 9cb87fec (6d146dfc) 6d146dfd
   5: ba43a338 (9cb87fec) 9cb87fed
   6: a0c394ff (ba43a338) ba43a339
   7: f72041ee (a0c394ff) a0c39500
   ...
\end{Verbatim}

\end{frame}
\end{task}

%-------------------------------------------------------------------------
\begin{frame}{Line Tracing vs. VCD Dumping}
%-------------------------------------------------------------------------

\begin{cbxlist}

  \1 \BF{Line Tracing}

    \2 Shows a single cycle per line and uses text characters to indicate
       state and how data moves through a system

    \vspace{0.08in}
    \2 Provides a way to visualize the high-level behavior of a system
       (e.g., pipeline diagrams, transaction diagrams)

    \vspace{0.08in}
    \2 Enables quickly debugging high-level functionality and performance
       bugs at the commandline

    \vspace{0.08in}
    \2 Can be used for FL, CL, and RTL models

  \1 \BF{VCD Dumping}

    \2 Captures the bit-level activity of every signal on every cycle

    \2 Requires a separate waveform viewer to visualize the signals

    \2 Provides a much more detailed view of a design

    \2 Mostly used for RTL models

\end{cbxlist}
\end{frame}

%-------------------------------------------------------------------------
\begin{task}\begin{frame}[fragile]{Simulate RegIncr with VCD dumping}
%-------------------------------------------------------------------------

\vspace{-0.15in}
\begin{Verbatim}[commandchars=\\\{\}]
 % cd \midtilde/pymtl-tut/build
 % python ../regincr/reg-incr-sim.py 10 --dump-vcd
 % gtkwave ./reg-incr-rtl-10.vcd
\end{Verbatim}

  \smallskip
  \cbxfigc{gtkwave-regincr.png}

\end{frame}
\end{task}

\begin{frame}{\IT{Hands-On:} PyMTL Basics with Max/RegIncr}
\begin{cbxlist}
  \1 Task 2.1: Experiment with \TT{Bits}
  \1 Task 2.2: Interactively simulate a max unit
  \1 Task 2.3: Write a registered incrementer (RegIncr) model
  \1 Task 2.4: Test the RegIncr model
  \1 Task 2.5: Translate the RegIncr model into Verilog
  \1 Task 2.6: Simulate the RegIncr model with line tracing
  \1 Task 2.7: Simulate the RegIncr model with VCD dumping
  \1 \BF{Task 2.8: Compose a pipeline with two RegIncr models}
  \1 \BF{Task 2.9: Compose a pipeline with N RegIncr models}
  \1 \BF{Task 2.10: Parameterize test to verify multiple Ns}
\end{cbxlist}
\end{frame}

%-------------------------------------------------------------------------
\begin{frame}{Structural Composition in PyMTL}
%-------------------------------------------------------------------------

\medskip
\begin{cbxcols}
\begin{column}{0.9\tw}
\begin{cbxlist}

  \1 In PyMTL, more complex designs can be created by hierarchically
     composing models using structural composition

  \1 Models are structurally composed by connecting their ports
     using \TT{s.connect()} or \TT{s.connect\_pairs()} statements

  \1 Data is communicated between PyMTL models using \TT{InPorts}
     and \TT{OutPorts}, not via method calls!

\end{cbxlist}
\end{column}
\end{cbxcols}
\end{frame}

%-------------------------------------------------------------------------
\begin{task}\begin{frame}[fragile]{Compose a pipeline with two RegIncrs}
%-------------------------------------------------------------------------

\vspace{-0.15in}
\begin{Verbatim}[commandchars=\\\{\}]
 % cd    \midtilde/pymtl-tut/build
 % gedit ../regincr/RegIncrPipeline.py
\end{Verbatim}
\vspace{-0.17in}

\lstinputlisting[xleftmargin={0.38in},firstline=9,lastline=21,firstnumber=9]%
{../../regincr_soln/RegIncrPipeline.py}

\vspace{-0.2in}
\begin{Verbatim}
 % py.test ../regincr/RegIncrPipeline_test.py -sv
\end{Verbatim}

\end{frame}
\end{task}

\begin{frame}[fragile]{Line Tracing from Pipelined RegIncr}

\begin{Verbatim}
  ../regincr_soln/RegIncrPipeline_test.py::test_simple
    0: 04 (00 00) 00
    1: 06 (05 02) 02
    2: 02 (07 06) 06
    3: 0f (03 08) 08
    4: 08 (10 04) 04
    5: 00 (09 11) 11
    6: 0a (01 0a) 0a
    7: 0a (0b 02) 02
    8: 0a (0b 0c) 0c
  PASSED
\end{Verbatim}

\end{frame}

%-------------------------------------------------------------------------
\begin{frame}{Parameterizing Models in PyMTL}
%-------------------------------------------------------------------------

\medskip
\begin{cbxcols}
\begin{column}{0.9\tw}
\begin{cbxlist}

  \1 Static elaboration code (everything inside \TT{\_\_init\_\_} that is
     not in a decorated function) can use the full expressiveness of
     Python

  \1 Static elaboration code constructs a connectivity graph of
     components, is always Verilog translatable \\
     (as long as leaf modules are translatable)

  \1 Enables the creation of powerful and highly-parameterizable
     hardware generators

\end{cbxlist}
\end{column}
\end{cbxcols}
\end{frame}

%-------------------------------------------------------------------------
\begin{task}\begin{frame}[fragile]{Compose a pipeline with N RegIncrs}
%-------------------------------------------------------------------------

\vspace{-0.15in}
\begin{Verbatim}[commandchars=\\\{\}]
 % cd \midtilde/pymtl-tut/build
 % gedit ../regincr/RegIncrParamPipeline.py
\end{Verbatim}
\vspace{-0.17in}

\lstinputlisting[xleftmargin={0.38in},firstline=9,lastline=22,firstnumber=9]%
{../../regincr/RegIncrParamPipeline.py}

\vspace{-0.2in}
\begin{Verbatim}
 % py.test ../regincr/RegIncrParamPipeline_test.py -sv
\end{Verbatim}

\end{frame}
\end{task}

%-------------------------------------------------------------------------
\begin{frame}{Parameterizing Tests in PyMTL}
%-------------------------------------------------------------------------

\medskip
\begin{cbxcols}
\begin{column}{0.9\tw}
\begin{cbxlist}

  \1 We leverage the opensource \TT{py.test} package to drive test
     collection and execution in PyMTL

  \1 Significantly simplifies process of writing unit tests, and enables
     functionality such as parallel/distributed test execution and
     coverage reporting via plugins

  \1 More importantly, \TT{py.test} has powerful facilities for writing
     extensive and highly parameterizable unit tests

  \1 One example: the \TT{@pytest.mark.parametrize} decorator

\end{cbxlist}
\end{column}
\end{cbxcols}
\end{frame}

%-------------------------------------------------------------------------
\begin{task}\begin{frame}[fragile]{Parameterize test to verify multiple Ns}
%-------------------------------------------------------------------------

\vspace{-0.15in}
\begin{Verbatim}[commandchars=\\\{\}]
 % cd \midtilde/pymtl-tut/build
 % gedit ../regincr/RegIncrParamPipeline_test.py
\end{Verbatim}

\lstinputlisting[xleftmargin={0.38in},firstline=43,lastline=51,firstnumber=43]%
{../../regincr_soln/RegIncrParamPipeline_test.py}

\begin{Verbatim}
 % py.test ../regincr/RegIncrParamPipeline_test.py -sv
\end{Verbatim}

\end{frame}
\end{task}

\begin{frame}[fragile]{Line Tracing from Pipelined RegIncr}

\begin{Verbatim}
../regincr_soln/RegIncrParamPipeline_test.py::test_simple[5]
  0: 04 (00 00 00 00 00) 00
  1: 06 (05 02 02 02 02) 02
  2: 02 (07 06 03 03 03) 03
  3: 0f (03 08 07 04 04) 04
  4: 08 (10 04 09 08 05) 05
  5: 00 (09 11 05 0a 09) 09
  6: 0a (01 0a 12 06 0b) 0b
  7: 0e (0b 02 0b 13 07) 07
  8: 10 (0f 0c 03 0c 14) 14
  9: 0c (11 10 0d 04 0d) 0d
  ...
PASSED
\end{Verbatim}

\end{frame}



\section[]{}
\scheduleslide{7}

%=========================================================================
% Presentation: ML Modeling
%=========================================================================

\section[{\it Presentation} Multi-Level Modeling]{}
\scheduleslide{7}

%-------------------------------------------------------------------------
\begin{frame}{Multi-Level Modeling}
%-------------------------------------------------------------------------
  \insertslide<1>{pymtl-intro}{5}
  \insertslide<2>{pymtl-intro}{4}
\end{frame}

%-------------------------------------------------------------------------
\begin{frame}{Multi-Level Modeling in PyMTL}
%-------------------------------------------------------------------------
\begin{itemize}
  \item FL modeling allows for the rapid creation of a working model.
        Designers can quickly experiment with interfaces and protocols.
  \item This design is \IT{manually refined} into a PyMTL CL model that
        includes timing, which is useful for rapid design space exploration.
  \item Promising architectures can again be \IT{manually refined} into a
        PyMTL RTL implementation to accurately model resources.
\vspace{0.1in}
\end{itemize}
  \cbxfigc{../images/pymtl-tut-refine.pdf}
\end{frame}

%-------------------------------------------------------------------------
\begin{frame}{Multi-Level Modeling in PyMTL}
%-------------------------------------------------------------------------
\begin{itemize}
  \item Verilog generated from PyMTL RTL can be passed to an
        EDA toolflow for accurate area, energy, and timing estimates.
  \item Throughout this process, the same PyMTL test harnesses can used to
        verify each model!
  \item Requires good design, the use of latency-insensitive interfaces
        helps considerably.
\vspace{0.1in}
\end{itemize}
  \cbxfigc{../images/pymtl-tut-refine.pdf}
\end{frame}

%-------------------------------------------------------------------------
\begin{frame}{FL Model in PyMTL}
%-------------------------------------------------------------------------
  \insertslide{ml-intro}{1}
\end{frame}

%-------------------------------------------------------------------------
\begin{frame}{CL Model in PyMTL}
%-------------------------------------------------------------------------
  \insertslide{ml-intro}{2}
\end{frame}

%-------------------------------------------------------------------------
\begin{frame}{RTL Model in PyMTL}
%-------------------------------------------------------------------------
  \insertslide{ml-intro}{3}
\end{frame}

%=========================================================================
% Hands-On: GCD Unit
%=========================================================================

\stepcounter{taskseccount}
\section[{\it Hands-On} GCD Unit]{}
\scheduleslide{9}

%-------------------------------------------------------------------------
\begin{frame}{PyMTL 102: The GCD Unit}
%-------------------------------------------------------------------------
\begin{itemize}
  \item Computes the greatest-common divisor of two numbers.
  \smallskip
  \item Uses a latency insensitive input protocol to accept messages only
        when sender has data available and GCD unit is ready.
  \smallskip
  \item Uses a latency insensitive output protocol to send results only
        when result is done and receiver is ready.
\end{itemize}

  \cbxfigc{tut3-gcd-fl.svg.pdf}
\end{frame}

%-------------------------------------------------------------------------
\begin{frame}{PyMTL 102: Bundled Interfaces}
%-------------------------------------------------------------------------
\vspace{-0.18in}
\begin{itemize}
  \item \BF{PortBundles} are used to simplify the handling of multi-signal
        interfaces, such as ValRdy:
\end{itemize}

\vspace{-0.15in}
\lstinputlisting[xleftmargin={0.38in},firstline=14,lastline=27,numbers=none]
{../../slide_examples/Bundles.py}

\end{frame}

%-------------------------------------------------------------------------
\begin{frame}{PyMTL 102: Complex Datatypes}
%-------------------------------------------------------------------------
\vspace{-0.18in}
\begin{itemize}
  \item \BF{BitStructs} are used to simplify communicating and interacting
        with complex packages of data:
\end{itemize}

\vspace{-0.15in}
\lstinputlisting[xleftmargin={0.38in},firstline=9,lastline=22,numbers=none]
{../../slide_examples/Structs_test.py}

\end{frame}

%-------------------------------------------------------------------------
\begin{frame}{PyMTL 102: Complex Datatypes}
%-------------------------------------------------------------------------

The GCD request message can be implemented as a BitStruct that has two
fields, one for each operand:

\vspace{.2in}
\cbxfigc{tut3-gcd-bitstruct.svg.pdf}
\end{frame}

%-------------------------------------------------------------------------
\begin{task}\begin{frame}[fragile]{Create a BitStruct for the GCD request}
%-------------------------------------------------------------------------
\vspace{-0.25in}
\begin{verbatim}
  % cd    ~/pymtl-tutorial/build
  % gedit ../gcd/GcdUnitMsg.py
\end{verbatim}

\lstinputlisting[xleftmargin={0.38in},firstline=12,lastline=19,firstnumber=12]%
{../../gcd_soln/GcdUnitMsg.py}

\begin{verbatim}
  % py.test ../gcd/GcdUnitMsg_test.py -v
\end{verbatim}
\end{frame}
\end{task}

%-------------------------------------------------------------------------
\begin{frame}{PyMTL 102: Latency Insensitive FL Models}
%-------------------------------------------------------------------------
\begin{itemize}
  \item Implementing latency insensitive communication protocols can be
        complex to implement and a challenge to debug.
  \smallskip
  \item PyMTL provides \BF{Interface Adapters} which abstract away the
        complexities of ValRdy, and expose simplified method interfaces.
\end{itemize}

  \begin{onlyenv}<1>
    \cbxfigc<1>{tut3-gcd-queues.svg.pdf}
  \end{onlyenv}

  \begin{onlyenv}<2>
  \vspace{-0.15in}
  \lstinputlisting[xleftmargin={0.38in},firstline=19,lastline=27,firstnumber=19]%
  {../../gcd/GCDUnitFL.py}
  \end{onlyenv}

\end{frame}

%-------------------------------------------------------------------------
\begin{task}\begin{frame}[fragile]{Build an FL model for the GCD unit}
%-------------------------------------------------------------------------
\vspace{-0.25in}
\begin{verbatim}
  % cd    ~/pymtl-tutorial/build
  % gedit ../gcd/GcdUnitFL.py
\end{verbatim}

\lstinputlisting[xleftmargin={0.38in},firstline=31,lastline=41,firstnumber=31]%
{../../gcd_soln/GcdUnitFL.py}

\begin{verbatim}
  % py.test ../gcd/GcdUnitFL_test.py -v
\end{verbatim}
\end{frame}
\end{task}

%-------------------------------------------------------------------------
\begin{frame}{PyMTL 102: Testing Latency Insensitive Models}
%-------------------------------------------------------------------------

\begin{itemize}
  \item To simplify testing of latency insensitive designs, PyMTL provides
        TestSources and TestSinks with ValRdy interfaces.
  \smallskip
  \item TestSources/TestSinks only transmit/accept data when the ``design
        under test'' is ready/valid.
  \smallskip
  \item Can be configured to insert random delays into valid/ready signals
        to verify latency insensitivity under various conditions.
\end{itemize}

  \cbxfigc{tut3-gcd-srcsink.svg.pdf}
\end{frame}

%-------------------------------------------------------------------------
\begin{task}\begin{frame}[fragile]{Create a latency insensitive test}
%-------------------------------------------------------------------------
\vspace{-0.25in}
\begin{verbatim}
  % cd    ~/pymtl-tutorial/build
  % gedit ../gcd/GcdUnitFL_simple_test.py
\end{verbatim}

\lstinputlisting[xleftmargin={0.38in},firstline=22,lastline=31,firstnumber=22]%
{../../gcd_soln/GcdUnitFL_simple_test.py}

\begin{verbatim}
  % py.test ../gcd/GcdUnitFL_simple_test.py -v -s
\end{verbatim}
\end{frame}
\end{task}

%-------------------------------------------------------------------------
\begin{frame}{PyMTL 102: Latency Insensitive CL Models}
%-------------------------------------------------------------------------
\vspace{-0.25in}
\lstinputlisting[xleftmargin={0.38in},firstline=1,lastline=16,numbers=none]%
{../../gcd_soln/fl_trace.out}

\end{frame}

%-------------------------------------------------------------------------
\begin{frame}{PyMTL 102: Latency Insensitive CL Models}
%-------------------------------------------------------------------------

\begin{itemize}
  \item Cycle-level models add timing information to a functional model
        and can provide a cycle-approximate estimation of performance.
  \smallskip
  \item Useful for rapid, initial exploration of an
        architectural design space.
  \smallskip
  \item We'll use a simple GCD algorithm to provide timing info.
\end{itemize}

  \cbxfigc{tut3-gcd-cl.svg.pdf}
\end{frame}

%-------------------------------------------------------------------------
\begin{task}\begin{frame}[fragile]{Add timing to the GCD CL model}
%-------------------------------------------------------------------------
\vspace{-0.25in}
\begin{verbatim}
  % cd    ~/pymtl-tutorial/build
  % py.test ../gcd/GcdUnitCL_test.py
  % py.test ../gcd/GcdUnitCL_test.py -k basic_0x0 -sv
  % gedit ../gcd/GcdUnitCL.py
\end{verbatim}

\vspace{-0.1in}
\lstinputlisting[xleftmargin={0.38in},firstline=17,lastline=25,firstnumber=17]%
{../../gcd_soln/GcdUnitCL.py}

\begin{verbatim}
  % py.test ../gcd/GcdUnitCL_test.py -k basic_0x0 -sv
\end{verbatim}
\end{frame}
\end{task}

%-------------------------------------------------------------------------
\begin{frame}{PyMTL 102: Latency Insensitive CL Models}
%-------------------------------------------------------------------------
\vspace{-0.25in}
\lstinputlisting[xleftmargin={0.38in},firstline=1,lastline=16,numbers=none]%
{../../gcd_soln/cl_trace.out}

\end{frame}

%-------------------------------------------------------------------------
\begin{frame}{PyMTL 102: Latency Insensitive RTL Models}
%-------------------------------------------------------------------------

\begin{itemize}
  \item RTL models allow us to accurately estimate executed cycles,
        cycle-time, area and energy when used with an EDA toolflow.
  \smallskip
  \item Constructing is time consuming! PyMTL tries to make it it more
        productive by providing a better design and testing environment.
\end{itemize}

  \begin{cbxcols}
  \begin{column}{0.5\tw}
  \cbxfigc{tut3-gcd-ctrl.svg.pdf}
  \end{column}
  \begin{column}{0.5\tw}
  \cbxfigc{tut3-gcd-dpath.svg.pdf}
  \end{column}
  \end{cbxcols}
\end{frame}

%-------------------------------------------------------------------------
\begin{frame}{PyMTL 102: Latency Insensitive RTL Models}
%-------------------------------------------------------------------------

\begin{itemize}
  \item Latency insensitive hardware generally separates logic into
        control and datapath (shown below).
  \smallskip
  \item Today, we won’t be writing RTL for GCD, but we’ll be fixing a bug
        in the RTL implementation of the state machine.
\end{itemize}

  \begin{cbxcols}
  \begin{column}{0.5\tw}
  \cbxfigc{tut3-gcd-ctrl-broken.svg.pdf}
  \end{column}
  \begin{column}{0.5\tw}
  \cbxfigc{tut3-gcd-dpath.svg.pdf}
  \end{column}
  \end{cbxcols}
\end{frame}


%-------------------------------------------------------------------------
\begin{task}\begin{frame}[fragile]{Fix the bug in the GCD RTL model}
%-------------------------------------------------------------------------
\vspace{-0.25in}
\begin{verbatim}
  % cd    ~/pymtl-tutorial/build
  % py.test ../gcd/GcdUnitRTL_test.py -k basic_0x0 -sv
  % gedit ../gcd/GcdUnitRTL.py
\end{verbatim}

\vspace{-0.2in}
\lstinputlisting[xleftmargin={0.38in},firstline=183,lastline=194,firstnumber=183]%
{../../gcd/GcdUnitRTL.py}

\vspace{-0.3in}
\begin{verbatim}
  % py.test ../gcd/GcdUnitRTL_test.py -k basic_0x0 -sv
\end{verbatim}
\end{frame}
\end{task}

%-------------------------------------------------------------------------
\begin{frame}{PyMTL 102: Latency Insensitive RTL Models}
%-------------------------------------------------------------------------
\vspace{-0.25in}
\lstinputlisting[xleftmargin={0.38in},firstline=1,lastline=16,numbers=none]%
{../../gcd_soln/rtl_trace.out}

\end{frame}

%-------------------------------------------------------------------------
\begin{task}\begin{frame}[fragile]{Verify generated Verilog GCD RTL}
%-------------------------------------------------------------------------
\vspace{-0.25in}
\begin{verbatim}
  % cd    ~/pymtl-tutorial/build
  % py.test ../gcd/GcdUnitRTL_test.py --test-verilog -sv
  % gedit GcdUnitRTL_0x791afe0d4d8c.v
\end{verbatim}

\vspace{-0.3in}
\lstinputlisting[xleftmargin={0.38in},firstline=6,lastline=19,firstnumber=4]%
{../../gcd_soln/GcdUnitRTL_0x791afe0d4d8c.v}
\end{frame}
\end{task}

%-------------------------------------------------------------------------
\begin{task}\begin{frame}[fragile]{Experiment with the GCD simulator}
%-------------------------------------------------------------------------
\vspace{-0.25in}
\begin{verbatim}
  # Simulating both the CL and RTL models

  % cd    ~/pymtl-tutorial/build
  % ../gcd/gcd-sim --stats --impl --cl  --input random
  % ../gcd/gcd-sim --stats --impl --rtl --input random

  # Experimenting with various datasets

  % ../gcd/gcd-sim --impl rtl --input random --trace
  % ../gcd/gcd-sim --impl rtl --input small  --trace
  % ../gcd/gcd-sim --impl rtl --input zeros  --trace

\end{verbatim}
\end{frame}
\end{task}

%-------------------------------------------------------------------------
\begin{frame}{PyMTL Next Steps and More Resources}
%-------------------------------------------------------------------------
Next Steps:
\begin{itemize}
  \item See the detailed tutorial on the Cornel ECE5745 website:
        \footnotesize{http://www.csl.cornell.edu/courses/ece5745/handouts/ece5745-tut-pymtl.pdf}
\end{itemize}

\smallskip
Check out the \BF{/docs} directory in the PyMTL repo for guides on:
\begin{itemize}
  \item Writing Pythonic PyMTL Models and Tests
  \item Writing Verilog Translatable PyMTL RTL
  \item Importing Verilog Components into PyMTL
\end{itemize}

\smallskip
Become a contributor!
\begin{itemize}
  \item https://github.com/cornell-brg/pymtl
  \item https://github.com/cornell-brg/pydgin
\end{itemize}
\end{frame}

%-------------------------------------------------------------------------
\begin{frame}{Thank you for coming!}
%-------------------------------------------------------------------------
  \begin{cbxcols}
  \begin{column}{0.5\tw}
  \vspace{.2in}
  \cbxfigc{pymtl-flat.svg.pdf}

  \vspace{.3in}
  \centering{PyMTL: A Unified Framework for Vertically Integrated Computer
  Architecture Research}

  \vspace{.363in}
  \centering{[ MICRO 2014 ]}
  \centering{\footnotesize{https://github.com/cornell-brg/pymtl}}
  \end{column}

  \begin{column}{0.5\tw}
  \vspace{.2in}
  \cbxfigc{pydgin-flat.svg.pdf}

  \vspace{.3in}
  \centering{Pydgin: Generating Fast Instruction Set Simulators from
  Simple Architecture Descriptions with Meta-Tracing JIT Compilers}

  \vspace{.2in}
  \centering{[ ISPASS 2015 ]}
  \centering{\footnotesize{https://github.com/cornell-brg/pydgin}}

  \end{column}
  \end{cbxcols}
\end{frame}


\section[]{}

%-------------------------------------------------------------------------
\begin{frame}{Thank you for coming!}
%-------------------------------------------------------------------------
  \begin{cbxcols}
  \begin{column}{0.5\tw}
  \vspace{.2in}
  \cbxfigc{pymtl-flat.svg.pdf}

  \vspace{.3in}
  \centering{PyMTL: A Unified Framework for Vertically Integrated Computer
  Architecture Research}

  \vspace{.363in}
  \centering{[ MICRO 2014 ]}
  \centering{\footnotesize{https://github.com/cornell-brg/pymtl}}
  \end{column}

  \begin{column}{0.5\tw}
  \vspace{.2in}
  \cbxfigc{pydgin-flat.svg.pdf}

  \vspace{.3in}
  \centering{Pydgin: Generating Fast Instruction Set Simulators from
  Simple Architecture Descriptions with Meta-Tracing JIT Compilers}

  \vspace{.2in}
  \centering{[ ISPASS 2015 ]}
  \centering{\footnotesize{https://github.com/cornell-brg/pydgin}}

  \end{column}
  \end{cbxcols}
\end{frame}

\end{document}

