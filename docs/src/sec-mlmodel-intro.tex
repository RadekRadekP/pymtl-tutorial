%=========================================================================
% Presentation: ML Modeling
%=========================================================================

\section[{\it Presentation} Multi-Level Modeling]{}
\scheduleslide{7}

%-------------------------------------------------------------------------
\begin{frame}{Multi-Level Modeling}
%-------------------------------------------------------------------------
  \insertslide<1>{pymtl-intro}{5}
  \insertslide<2>{pymtl-intro}{4}
\end{frame}

%-------------------------------------------------------------------------
\begin{frame}{Multi-Level Modeling in PyMTL}
%-------------------------------------------------------------------------
\begin{itemize}
  \item FL modeling allows for the rapid creation of a working model.
        Designers can quickly experiment with interfaces and protocols.
  \item This design is \IT{manually refined} into a PyMTL CL model that
        includes timing, which is useful for rapid design space exploration.
  \item Promising architectures can again be \IT{manually refined} into a
        PyMTL RTL implementation to accurately model resources.
\vspace{0.1in}
\end{itemize}
  \cbxfigc{../images/pymtl-tut-refine.pdf}
\end{frame}

%-------------------------------------------------------------------------
\begin{frame}{Multi-Level Modeling in PyMTL}
%-------------------------------------------------------------------------
\begin{itemize}
  \item Verilog generated from PyMTL RTL can be passed to an
        EDA toolflow for accurate area, energy, and timing estimates.
  \item Throughout this process, the same PyMTL test harnesses can used to
        verify each model!
  \item Requires good design, the use of latency-insensitive interfaces
        helps considerably.
\vspace{0.1in}
\end{itemize}
  \cbxfigc{../images/pymtl-tut-refine.pdf}
\end{frame}

%-------------------------------------------------------------------------
\begin{frame}{FL Model in PyMTL}
%-------------------------------------------------------------------------
  \insertslide{ml-intro}{1}
\end{frame}

%-------------------------------------------------------------------------
\begin{frame}{CL Model in PyMTL}
%-------------------------------------------------------------------------
  \insertslide{ml-intro}{2}
\end{frame}

%-------------------------------------------------------------------------
\begin{frame}{RTL Model in PyMTL}
%-------------------------------------------------------------------------
  \insertslide{ml-intro}{3}
\end{frame}
