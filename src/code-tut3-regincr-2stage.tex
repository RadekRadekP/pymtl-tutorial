%=========================================================================
% code-tut3-regincr-2stage
%=========================================================================

\begin{figure}

  \begin{lstlisting}[xleftmargin={0.9in}]
#=========================================================================
# RegIncr2stage
#=========================================================================
# Two-stage registered incrementer that uses structural composition to
# instantiate and connect two instances of the single-stage registered
# incrementer.

from pymtl   import *
from RegIncr import RegIncr

class RegIncr2stage( Model ):

  # Constructor

  def __init__( s ):

    # Port-based interface

    s.in_ = InPort  ( Bits(8) )
    s.out = OutPort ( Bits(8) )

    # First stage

    s.reg_incr_0 = RegIncr()

    s.connect( s.in_, s.reg_incr_0.in_ )

    # Second stage

    s.reg_incr_1 = RegIncr()

    s.connect( s.reg_incr_0.out, s.reg_incr_1.in_ )
    s.connect( s.reg_incr_1.out, s.out )

  # Line Tracing

  def line_trace( s ):
    return "{} ({}|{}) {}".format(
      s.in_,
      s.reg_incr_0.line_trace(),
      s.reg_incr_1.line_trace(),
      s.out
    )
\end{lstlisting}

  \centerline{\small Code at
    \url{https://github.com/cbatten/y/blob/master/RegIncr2stage.py}}

  \caption{\textbf{Two-Stage Registered Incrementer --} An eight-bit
    two-stage registered incrementer corresponding to
    Figure~\ref{fig-tut3-regincr-2stage}. This model is implemented using
    structural composition to instantiate and connect two instances of the
    single-stage register incrementer.}
  \label{code-tut3-regincr-2stage}

\end{figure}

