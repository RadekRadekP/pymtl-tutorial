%=========================================================================
% code-tut3-sort-rtl-struct
%=========================================================================

\begin{figure}[b]

  \begin{lstlisting}[xleftmargin={0.9in}]
#=========================================================================
# SortUnitStructRTL
#=========================================================================

from pymtl          import *
from pclib.rtl.regs import Reg, RegRst
from MinMaxUnit     import MinMaxUnit

class SortUnitStructRTL( Model ):

  def __init__( s, nbits=8 ):

    #---------------------------------------------------------------------
    # Interface
    #---------------------------------------------------------------------

    s.in_val  = InPort (1)
    s.in_     = InPort [4](nbits)

    s.out_val = OutPort(1)
    s.out     = OutPort[4](nbits)

    #---------------------------------------------------------------------
    # Stage S0->S1 pipeline registers
    #---------------------------------------------------------------------

    s.val_S0S1 = RegRst(1)
    s.elm_S0S1 = Reg[4](nbits)

    s.connect( s.in_val, s.val_S0S1.in_ )
    for i in xrange(4):
      s.connect( s.in_[i], s.elm_S0S1[i].in_ )

    #---------------------------------------------------------------------
    # Stage S1 combinational logic
    #---------------------------------------------------------------------

    s.minmax0_S1 = MinMax(nbits)

    s.connect( s.elm_S0S1[0].out, s.minmax0_S1.in0 )
    s.connect( s.elm_S0S1[1].out, s.minmax0_S1.in1 )

    s.minmax1_S1 = MinMax(nbits)

    s.connect( s.elm_S0S1[2].out, s.minmax1_S1.in0 )
    s.connect( s.elm_S0S1[3].out, s.minmax1_S1.in1 )

    ...
\end{lstlisting}

  \caption{\textbf{Sort Unit Structural RTL Model --} RTL model of
    four-element sort unit corresponding to
    Figure~\ref{fig-tut3-sort-rtl}. For simplicity only the
    interface and first pipeline stage are shown.}
  \label{code-tut3-sort-rtl-struct}

\end{figure}

