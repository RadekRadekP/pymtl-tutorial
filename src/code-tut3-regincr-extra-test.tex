%=========================================================================
% code-tut3-regincr-extra-test
%=========================================================================

\begin{figure}

  \begin{lstlisting}[xleftmargin={0.9in}]
#=========================================================================
# RegIncr_extra_test
#=========================================================================

from pymtl      import *
from pclib.test import run_test_vector_sim
from RegIncr    import RegIncr

#-------------------------------------------------------------------------
# test_small
#-------------------------------------------------------------------------

def test_small( dump_vcd ):
  run_test_vector_sim( RegIncr(), [
    ('in_   out*'),
    [ 0x00, '?'  ],
    [ 0x03, 0x01 ],
    [ 0x06, 0x04 ],
    [ 0x00, 0x07 ],
  ], dump_vcd )

#-------------------------------------------------------------------------
# test_large
#-------------------------------------------------------------------------

def test_large( dump_vcd ):
  run_test_vector_sim( RegIncr(), [
    ('in_   out*'),
    [ 0xa0, '?'  ],
    [ 0xb3, 0xa1 ],
    [ 0xc6, 0xb4 ],
    [ 0x00, 0xc7 ],
  ], dump_vcd )

#-------------------------------------------------------------------------
# test_overflow
#-------------------------------------------------------------------------

def test_overflow( dump_vcd ):
  run_test_vector_sim( RegIncr(), [
    ('in_   out*'),
    [ 0x00, '?'  ],
    [ 0xfe, 0x01 ],
    [ 0xff, 0xff ],
    [ 0x00, 0x00 ],
  ], dump_vcd )
\end{lstlisting}

  \centerline{\small Code at
    \url{https://github.com/cbatten/y/blob/master/RegIncr_extra_test.py}}

  \caption{\textbf{Unit Test Script using Test Vectors for Registered
      Incrementer --} A unit test for the eight-bit registered
    incrementer in Figure~\ref{code-tut3-regincr}, which uses test
    vectors and the \TT{py.test} unit testing framework.}
  \label{code-tut3-regincr-extra-test}

\end{figure}

