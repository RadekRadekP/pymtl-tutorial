%=========================================================================
% ECE 4750 Tutorial 3 : Verilog
%=========================================================================

\documentclass{cbxdoc}

% Automatix LaTeX build system modules

\usepackage{svg}

%-------------------------------------------------------------------------
% Document Parameters
%-------------------------------------------------------------------------

\title{PyMTL Cheat Sheet}

%-------------------------------------------------------------------------
% Document
%-------------------------------------------------------------------------

\begin{document}
%=========================================================================
% pymtl-listings
%=========================================================================

\definecolor{dmlgreen}    {RGB}{51,  160,  44}
\definecolor{dmlblue}     {RGB}{31,  120, 180}
\definecolor{dmlred}      {RGB}{202,   0,  32}

\lstdefinestyle{simple}{%
  language=Python,
  numbers={none},
  basicstyle={\ttfamily},
  moredelim={[is][\underbar]{__}{__}}
}

\lstset
{%
  language=Python,%
  alsoletter={.},
  morekeywords=[2]{
    @pytest.mark.parametrize,
    @s.tick,
    @s.tick_fl,
    @s.tick_cl,
    @s.tick_rtl,
    @s.combinational,
    s.connect,
    s.connect_auto,
    s.connect_dict,
    s.connect_pairs,
    Bits,
    Wire,
    InPort,
    OutPort,
    SimulationTool,
    TranslationTool,
    Model
  },
  basicstyle={\ttfamily\footnotesize},%
  keywordstyle={\color{cbxgreenC}},%
  keywordstyle={[2]\color{cbxblueC}},%
  commentstyle={\color{cbxredC}},
  lineskip={-0.005in},%
  numbers={left},%
  numberstyle={\tiny},%
  xleftmargin={0.2in},%
  showstringspaces={false},%
  keepspaces={true},%
  upquote={true},%
  columns={fullflexible},%
  stringstyle={\color{brown}},%
}%



\maketitle
\bigskip

%-------------------------------------------------------------------------
% PyMTL Model Template
%-------------------------------------------------------------------------

\section{PyMTL Model Template}

\begin{verbatim}
  class MyModel( Model ):
    def __init__( s, dtype ):

      # port declarations

      s.in_ = InPort( dtype )
      s.out = [ OutPort( dtype ) for _ in range( 4 ) ]

      # submodule declaration and connections

      s.mod = SubMod( dtype, other, params )

      s.connect( s.in_, s.mod.in_ )
      s.connect_pairs(
        s.mod.out[0], s.out[0],
        s.mod.out[1], s.out[1],
      )

      # concurrent logic specifications

      @s.combinational
      def comb_logic():
        s.out[2].value = s.in_

      @s.tick_rtl
      def seq_logic():
        s.out[3].next = s.in_

      # wire declarations

      s.this_out = Wire( dtype )
      s.last_out = Wire( dtype )

      @s.combinational
      def comb_logic2():
        s.this_out.value = s.in_

      @s.tick_rtl
      def seq_logic2():
        if s.reset: s.last_out.next = 0
        else:       s.last_out.next = s.this_out
\end{verbatim}

\clearpage

%-------------------------------------------------------------------------
% PyMTL Model Guidelines
%-------------------------------------------------------------------------
\section{PyMTL Modeling Guidelines}
\begin{itemize} %\itemindent 20pt
  \item Data is communicated between PyMTL models using \TT{InPorts} \&
         \TT{OutPorts}, not method calls!\\
         Models are composed \BF{structurally} by connecting their ports
         using \TT{s.connect} statements.
  \item Instantiated \TT{InPort}, \TT{OutPort}, and \TT{Wire} objects must
         stored as members of the model class:
           \begin{lstlisting}[style=simple,gobble=10]
             __s.__signal_name = InPort(8) \end{lstlisting}%
  \item Instantiated submodels must also be stored as members of the model
         class:
           \begin{lstlisting}[style=simple,gobble=10]
             __s.__model_name = Incrementer( dtype=8, incr_amt=1 )\end{lstlisting}%
  \item \BF{Sequential} concurrent logic blocks must always be annotated
         with \TT{\underline{@s.tick\_(fl|cl|rtl)}}. \\
        \TT{InPorts}/\TT{OutPorts}/\TT{Wires} updated in \BF{sequential}
         blocks must write the \underline{\TT{.next}} attribute.
           \begin{lstlisting}[style=simple,gobble=10]
             @s.tick_rtl
             def sequential_block():
               s.out.__next__ = s.a & s.b \end{lstlisting}%
  \item \BF{Combinational} concurrent logic blocks must always be
         annotated with \TT{\underline{@s.combinational}}.
        \TT{InPorts}/\TT{OutPorts}/\TT{Wires} updated in \BF{combinational}
         blocks must write the \underline{\TT{.value}} attribute.
           \begin{lstlisting}[style=simple,gobble=10]
             @s.combinational
             def sequential_block():
               s.out.__value__ = s.a & s.b \end{lstlisting}%
\end{itemize}


%-------------------------------------------------------------------------
% Using the PyMTL SimulationTool
%-------------------------------------------------------------------------
\section{Using the PyMTL SimulationTool}
\begin{verbatim}
  model = MyModel( dtype = 8 ) 
  model.elaborate()
  sim = SimulationTool()

  for i in range( 10 ):
    model.in_.value = i
    sim.cycle()
    sim.print_line_trace()
\end{verbatim}

%-------------------------------------------------------------------------
% Using the PyMTL TranslationTool
%-------------------------------------------------------------------------
\section{Using the PyMTL TranslationTool}
\begin{verbatim}
  model = MyModel( dtype = 8 ) 
  model.elaborate()
  model = TranslationTool( model )
\end{verbatim}

%-------------------------------------------------------------------------
% Writing Powerful Unit Tests in py.test
%-------------------------------------------------------------------------
\section{Writing Powerful Unit Tests in py.test}
\begin{verbatim}
  @pytest.mark.parameterize(`dtype', 8, 32 )
  def test_my_model( test_verilog, dump_vcd, dtype ):
     model = MyModel( dtype )
     model.elaborate()
     model.vcd_file = dump_vcd
     if test_verilog:
       model = TranslationTool( model )
     sim = SimulationTool( model )

     print()
     for i in range( 10 ):
       data = random.randrange( 2**dtype )
       model.in_.value = data
       sim.cycle()
       sim.print_line_trace()
       assert model.out[2].value = data
\end{verbatim}

%-------------------------------------------------------------------------
% PyMTL Python Test
%-------------------------------------------------------------------------

\section{Running PyMTL Python Tests with py.test}

\begin{verbatim}
 % py.test ../my_model_dir --verbose --tb=line
 % py.test ../my_model_dir -k <grep_str>
 % py.test ../my_model_dir -k <grep_str> -s
\end{verbatim}

%-------------------------------------------------------------------------
% PyMTL Verilog Test
%-------------------------------------------------------------------------

\section{Running PyMTL Verilog Tests with py.test}

\begin{verbatim}
 % py.test ../my_model_dir --verbose --test-verilog
 % py.test ../my_model_dir --verbose --test-verilog --dump-vcd
\end{verbatim}

%-------------------------------------------------------------------------
% PyMTL Bits Operators
%-------------------------------------------------------------------------

\clearpage

\section{PyMTL Bits Operators}

\newenvironment{optbl}[3][0.32in]
{
  \vspace{0pt}\centering
  \BF{#2}\vphantom{g}
  \smallskip

  \begin{tabular}{>{\ttfamily\centering\arraybackslash}p{#1}p{#3}}
\toprule
}{
\bottomrule
  \end{tabular}
}

\begin{minipage}[t]{0.25\tw}

\begin{optbl}{Logical Operators}{0.6\tw}
  \verb|&|   & bitwise AND   \\
  |          & bitwise OR    \\
  \verb|^|   & bitwise XOR   \\
  \verb|^~|  & bitwise XNOR  \\
  \verb|~|   & bitwise NOT   \\
\end{optbl}

\smallskip

\begin{optbl}{Arithmetic Operators}{0.6\tw}
  \verb|+|   & addition         \\
  \verb|-|   & subtraction      \\
\end{optbl}

\end{minipage}
%
\hfill
%
\begin{minipage}[t]{0.32\tw}

\begin{optbl}[0.65in]{Reduction Operators}{0.52\tw}
  \verb|reduce_and| & reduce via AND \\
  \verb|reduce_or|  & reduce via OR  \\
  \verb|reduce_xor| & reduce via XOR \\
\end{optbl}

\smallskip

\begin{optbl}{Shift Operators}{0.693\tw}
  \verb|>>|  & shift right            \\
  \verb|<<|  & shift left             \\
\end{optbl}

\begin{optbl}{Slice Operators}{0.693\tw}
  \verb|[x]|   & get/set bit x            \\
  \verb|[x:y]| & get/set bits x upto y    \\
\end{optbl}

\end{minipage}
%
\hfill
%
\begin{minipage}[t]{0.335\tw}

\begin{optbl}{Relational Operators}{0.7\tw}
  \verb|==|  & equal                  \\
  \verb|!=|  & not equal              \\
  \verb|>|   & greater than           \\
  \verb|>=|  & greater than or equals \\
  \verb|<|   & less than              \\
  \verb|<=|  & less than or equals    \\
\end{optbl}

\smallskip

\begin{optbl}{Other Functions}{0.7\tw}
  \verb|concat| & concatenate         \\
  \verb|sext|   & sign-extension      \\
  \verb|zext|   & zero-extension      \\
\end{optbl}

\end{minipage}


\end{document}

