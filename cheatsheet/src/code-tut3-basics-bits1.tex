%=========================================================================
% code-tut3-basics-bits1
%=========================================================================

%\begin{figure}

  \begin{lstlisting}[gobble=4]
    # Bits constructor specifies bitwidth
    # and initial value

    >>> a = Bits( 16, 37 )
    >>> type(a)
    <class 'pymtl.datatypes.Bits.Bits'>
    >>> a
    Bits( 16, 0x0025 )
    >>> a = 47
    >>> type(a)
    <type 'int'>
    >>> a
    47

    # Getting number of bits and value

    >>> a = Bits( 16, 37 )
    >>> a.nbits
    16
    >>> a.uint()
    37

    # Using binary and hexadecimal literals

    >>> Bits( 8, 0b10101100 )
    Bits( 8, 0xac )
    >>> Bits( 32, 0xabcd0123 )
    Bits( 32, 0xabcd0123 )

    # Negative values stored in two's complement

    >>> Bits( 8, -1 )
    Bits( 8, 0xff )
    >>> Bits( 8, -2 )
    Bits( 8, 0xfe )
    >>> Bits( 8, -128 )
    Bits( 8, 0x80 )

    # Initial values that cannot be stored with
    # given bitwidth throw an exception

    >>> Bits( 8,  300 )
    >>> Bits( 8, -300 )

    # Truncating initial values

    >>> Bits( 8, 300, trunc=True )
    Bits( 8, 0x2c )
    >>> Bits( 8, 0xdeadbeef, trunc=True )
    Bits( 8, 0xef )
\end{lstlisting}

  \captionof{figure}{\textbf{Creating \TT{Bits} Objects}}
  \label{code-tut3-basics-bits1}

%\end{figure}
