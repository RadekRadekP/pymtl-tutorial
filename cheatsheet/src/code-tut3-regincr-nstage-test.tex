%=========================================================================
% code-tut3-regincr-nstage-test
%=========================================================================

\begin{figure}

  \begin{lstlisting}[xleftmargin={0.9in}]
#=========================================================================
# RegincrNstage_test
#=========================================================================

import collections
import pytest

from random        import sample

from pymtl         import *
from pclib.test    import run_test_vector_sim, mk_test_case_table
from RegIncrNstage import RegIncrNstage

#-------------------------------------------------------------------------
# mk_test_vector_table
#-------------------------------------------------------------------------

def mk_test_vector_table( nstages, inputs ):

  inputs.extend( [0]*nstages )

  test_vector_table = [ ('in_ out*') ]
  last_results = collections.deque( ['?']*nstages )
  for input_ in inputs:
    test_vector_table.append( [ input_, last_results.popleft() ] )
    last_results.append( Bits( 8, input_ + nstages, trunc=True ) )

  return test_vector_table

#-------------------------------------------------------------------------
# Parameterized Testing with Test Case Table
#-------------------------------------------------------------------------

test_case_table = mk_test_case_table([
  (                    "nstages inputs                "),
  [ "2stage_small",    2,       [ 0x00, 0x03, 0x06 ]   ],
  [ "2stage_large",    2,       [ 0xa0, 0xb3, 0xc6 ]   ],
  [ "2stage_overflow", 2,       [ 0x00, 0xfe, 0xff ]   ],
  [ "2stage_random",   2,       sample(range(0xff),20) ],
  [ "3stage_small",    3,       [ 0x00, 0x03, 0x06 ]   ],
  [ "3stage_large",    3,       [ 0xa0, 0xb3, 0xc6 ]   ],
  [ "3stage_overflow", 3,       [ 0x00, 0xfe, 0xff ]   ],
  [ "3stage_random",   3,       sample(range(0xff),20) ],
])

@pytest.mark.parametrize( **test_case_table )
def test( test_params, dump_vcd ):
  nstages = test_params.nstages
  inputs  = test_params.inputs
  run_test_vector_sim( RegIncrNstage( nstages ),
    mk_test_vector_table( nstages, inputs ), dump_vcd )

#-------------------------------------------------------------------------
# Parameterized Testing of With nstages = [ 1, 2, 3, 4, 5, 6 ]
#-------------------------------------------------------------------------

@pytest.mark.parametrize( "n", [ 1, 2, 3, 4, 5, 6 ] )
def test_random( n, dump_vcd ):
  run_test_vector_sim( RegIncrNstage( nstages=n ),
    mk_test_vector_table( n, sample(range(0xff),20) ), dump_vcd )
\end{lstlisting}

  \centerline{\small Code at
    \url{https://github.com/cbatten/y/blob/master/RegIncrNstage_test.py}}

  \caption{\textbf{Unit Test Script for Parameterized Registered
      Incrementer --} A unit test for the parameterized registered
    incrementer shown in Figure~\ref{code-tut3-regincr-nstage}.}
  \label{code-tut3-regincr-nstage-test}

\end{figure}

