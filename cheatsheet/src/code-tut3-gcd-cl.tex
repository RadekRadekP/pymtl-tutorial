%=========================================================================
% code-tut3-gcd-cl
%=========================================================================

\begin{figure}

  \begin{lstlisting}[xleftmargin={0.9in}]
#-------------------------------------------------------------------------
# GCD: algorithm and timing model
#-------------------------------------------------------------------------

def gcd( a, b ):

  ncycles = 0
  while True:
    ncycles += 1
    if a < b:
      a,b = b,a
    elif b != 0:
      a = a - b
    else:
      return (a,ncycles)

#-------------------------------------------------------------------------
# GcdUnitCL
#-------------------------------------------------------------------------

class GcdUnitCL( Model ):

  def __init__( s ):

    # Interface

    s.req    = InValRdyBundle  (32)
    s.resp   = OutValRdyBundle (16)

    # Adapters

    s.req_q  = InValRdyQueueAdapter  ( s.req  )
    s.resp_q = OutValRdyQueueAdapter ( s.resp )

    # Member variables

    s.result  = 0
    s.counter = 0

    # Concurrent block

    @s.tick_cl
    def block():

      # Tick the queue adapters

      s.req_q.xtick()
      s.resp_q.xtick()

      # Handle delay to model the gcd unit latency

      if s.counter > 0:
        s.counter -= 1
        if s.counter == 0:
          s.resp_q.enq( s.result )

      # If we have a new message and the output queue is not full

      elif not s.req_q.empty() and not s.resp_q.full():
        req_msg = s.req_q.deq()
        s.result,s.counter = gcd( req_msg[0:16], req_msg[16:32] )
\end{lstlisting}

  \caption{\textbf{Excerpt from Gcd Unit CL Model --} CL model of greatest-common
    divisor unit corresponding to Figure~\ref{fig-tut3-gcd-cl}.}
  \label{code-tut3-gcd-cl}

\end{figure}

